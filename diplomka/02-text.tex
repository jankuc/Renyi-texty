%\chapter*{Introduction}

\chapter{Minimum distance  estimation}
\section{Minimum distance estimator}
In this section we consider a parametric family $\mathcal{F} = \lbrace F_\theta : \theta \in \Theta \subset \mathbb{R}^m \rbrace$, of distribution functions on $\mathbb{R}$, where $\Theta \subset \mathbb{R}^m$ is metric space with Euclidean distance 
\begin{equation}
	\rho (\theta_1,\theta_2) = \sqrt{(\theta_1-\theta_2)(\theta_1-\theta_2)^T}, \quad \theta_1, \, \theta_2 \in \Theta.
\end{equation}
Furthermore, we assume identifiability of the family $\mathcal{F}$, which means 
\begin{equation}
\theta_1 \neq \theta_2 \Rightarrow F_{\theta_1} \neq F_{\theta_2}
\end{equation}
and also that all of $F_\theta(x), \, x \in \mathbb{R}$ are measurable. Our estimators will be calculated on the basis of i.i.d. observations $\mathbf{X}_n = (X_1, \ldots ,X_n)$ all distributed by distribution function $F_{\theta_0} \in \mathcal{F}$. Then $F_{\theta_0} \in \mathcal{F}$ is a true distribution function and $\theta_0 \in \Theta$ is a true parameter.

We are interested in non-parametric estimates of the unknown distribution $F_{\theta_0}$ based on $\mathbf{X}_n$, which are sequences of mappings $ \mathbb{R}^n \rightarrow \mathcal{F}(\mathbb{R}) $ or directly in estimates of the unknown parameter $\theta_0$ (point estimator) based on the same sample $\mathbf{X}_n$, which is sequence of functions $\\hat{\theta}_n : \mathbb{R}^n \rightarrow \Theta$. If $\mathcal{F}$ is dominated by a $\sigma$-finite measure $\lambda$ and we write \RN density as
\begin{equation}
f_\theta = \dfrac{\mathrm{d} F_\theta}{\mathrm{d} \lambda},
\end{equation}
we might be also interested in estimator $\hat{f}_n : \mathbb{R}^n \rightarrow \lbrace f_\theta : \theta \in \Theta  \rbrace$ of unknown density $f_{\theta_0}$.

The error of point estimates $\hat{\theta}_n$ is evaluated by $\rho (\hat{\theta}_n,\theta_0)$, error of density estimates by  $\| f_{\hat{\theta}_n} - f_{\theta_0} \|$, where
\begin{equation}
\| f_{\theta_1} - f_{\theta_2} \| = \int{|f_{\theta_1} - f_{\theta_2}|\,\mathrm{d}\lambda}
\end{equation}
is $\mathrm{L}_1$-norm. For evaluating the deviation of distribution functions we need some distance $\mathfrak{D}(F,G) $ on $\mathcal{F}(\mathbb{R}).$

\begin{definition} % 1 
For a set of random variables $X_n$ the notation $X_n = o_p(1)$ means that it holds $\lim_{n \rightarrow \infty } X_n = 0$ in probability. The notation $X_n = o_p(\varepsilon_n)$ means that $\frac{X_n}{\varepsilon_n} = o_p(1)$ for some sequence $\varepsilon_n \searrow 0$.
\end{definition} 

\begin{definition} % 2
We say that an estimator $\hat{F}_n$ is consistent estimator of $F_{\theta_0}$ in distance $\mathfrak{D}$, if $\mathfrak{D}(\hat{F}_n, F_{\theta_0}) = o_p (1)$ and is measurable in $\mathbf{X}_n$.
\end{definition}

%\begin{definition}
%Řekneme, že posloupnost náhodných veličin $\lbrace X_n \rbrace$ s distribučními funkcemi $\lbrace F_n \rbrace$ je omezená v pravděpodobnosti, pokud pro každé $\varepsilon > 0 $ existují $M$ a $N$ takové, že 
%	\begin{equation}
%		F_n(M) - F_n(-M) > 1-\varepsilon, \quad \forall n > N.
%	\end{equation}
%	Tuto skutečnost značíme $X_n = O_p(1)$.
%\end{definition} 

As an example of non-parametric estimator of distribution function we can use empirical distribution function defined as 
\begin{equation}
F_n(x)=\frac{1}{n}\sum_{j=1}^n \textbf{I}_{(-\infty,x]}(X_j), \qquad 
x\in\mathbb{R}.
\end{equation}

%
%\begin{theorem}[Glivenko-Cantelli]
%Nech\v{t} jsou $X_1,\dots,X_n$ stejně a nezávisle rozdělené reálné ná\-ho\-dné veličiny s distribuční funkcí $F$ a nechť $F_n(x)$ je empirická distribuční funkce. Pak pro každé $n\in \mathbb{N}$ a $\varepsilon>0$ platí
%
%\begin{equation}
%P\left(\sup_{x\in\mathbb{R}}|F(x)-F_n(x)|>\varepsilon\right) \leq 8(n+1)\exp{\left\{ \frac{-n{\varepsilon}^2}{32}\right\}}.
%\end{equation}
%Z Borel-Cantelliho lemma pak plyne
%\begin{equation}
%\lim_{n \rightarrow +\infty}\sup_{x \in \mathbb{R}}|F(x)-F_n(x)|=0  \quad \text{s. j.}
%\end{equation}
%\end{theorem}

Now we can define minimum distance estimator (MDE) and asymptotically minimum distance estimator (AMDE).

\begin{definition}\label{def-mde}
	Let $\mathfrak{D}(\cdot, \cdot) $ be distance on $\mathcal{F}$ and $\hat{F}_n$ an estimator of distribution function $F_{\theta_0}$. A point estimator $\hat{\theta}_n$ measurable in $\Theta$ satisfying the condition
	\begin{equation}
		\mathfrak{D}(\hat{F}_n, F_{\hat{\theta}_n}) = \inf_{\theta \in \Theta}(\hat{F}_n, F_{\theta})
		\label{eq:MDE}
	\end{equation}
	is said to be minimum distance estimator (MDE). If the family $\mathcal{F}$ is dominated by some $\sigma$-finite measure $ \lambda$, then estimator's \RN density $f_{\hat{\theta}_n}$ is said to be minimum distance probability density estimator. If the estimator meets at least condition
	\begin{equation}
		\mathfrak{D}(F_n, F_{\hat{\theta}_n}) - \inf_{\theta \in \Theta}(F_n, F_{\theta_0}) = o_p(n^{-\frac{1}{2}}), 
	\end{equation}
	 instead of condition \ref{eq:MDE}, the function $\hat{\theta}_n$ is said to be approximate minimum distance estimator (AMDE) and the corresponding probability density function $f_{\hat{\theta}_n}$ is said to be approximate minimum distance probability density estimator.
\end{definition}

Note that the rate $n^{-\frac{1}{2}}$ is not important and it can be replaced by any $\varepsilon_n$ tending to 0 sufficiently fast.

\section{Robustness}

Let $\mathcal{P} = \lbrace P_\theta : \theta \in \Theta \subset \mathbb{R}^m \rbrace$ is set of probability distributions on measurable space $\left(\mathcal{X},\mathcal{A}\right)$. Let $T: \mathcal{P} \rightarrow \mathbb{R}^m$ is fisher consistent functional, that means $T(P_\theta) = \theta$ for all $\theta \in \Theta$. We introduce convex mixtures of distributions.

\begin{definition}
	Let us have $\varepsilon \in [0,1]$ and $P, Q \in \mathcal{P}$. We denote convex mixture of distributions $P, Q$ with coefficient $\varepsilon$ by
	\begin{equation}
		P_\varepsilon(Q) = (1-\varepsilon)P + \varepsilon Q.
		\label{konvex-smes}
	\end{equation}
\end{definition}

\subsection{Influence  function}

Here we define influence function by which we measure effect of one measurement $x$ on estimate obtained by functional $T$.

\begin{definition}
	Let $\delta_x$ denotes Dirac delta function in $x,\, x \in \mathcal{X}$. Influence function $\IF{x}$ of functional $T$ in $P,\, P \in \mathcal{P}$ is then defined as
	\begin{equation}
		\IF{x} = \lim_{\varepsilon \rightarrow 0_+} \frac{T(P_\varepsilon(\delta_x)) - T(P)}{\varepsilon} = \lim_{\varepsilon \rightarrow 0_+} \frac{T((1-\varepsilon)P + \varepsilon\delta_x) - T(P)}{\varepsilon}.
	\end{equation} 
\end{definition}

We can see from the definition, that if $\IF{x}$ is not bounded then even one distant measurement can cause a total failure of the estimator $T$.

We introduce three measures of robustness from \cite{Antoch92}, which characterize some possible disruptions of our model. 

\begin{definition}
	By gross error sensitivity measure of functional $T$ for probability distribution $P$ we mean function  $\gamma^*$ defined as 
	\begin{equation}
		\gamma^* = \sup_{x \in \mathcal{X}} |\IF{x}|.
	\end{equation}
\end{definition}
This function gives a notion of the worst possible effect of occurrence of gross error in data sample on the value of estimator.  From the point of robustness of the estimator, it is therefore desirable for the value of $\gamma^*$ to be finite. Estimators for which the value of $\gamma^*$ is finite are called B-robust.

\begin{definition}
	By local shift sensitivity measure of functional $T$ for probability distribution $P$ we mean function  $\lambda^*$ defined as 
	\begin{equation}
			\lambda^* = \sup_{x,y \in \mathcal{X},x \neq y}  \left| \frac{\IF{y} - \IF{x}}{y-x} \right|.
	\end{equation}
\end{definition}

If the studied probability distribution $P$ is symmetrical around point $x=0$, we can define another measure of robustness.

\begin{definition}
	Rejection point $\rho^*$ for functional $T$ for probability distribution $P$ is defined as 
	\begin{equation}
			\rho^* = \inf_{x \in \mathcal{X}} \lbrace r>0 \, | \, \IF{x} = 0 \quad \text{ for all }\quad |x| > r\rbrace.
	\end{equation}
	If such constant $r$  does not exist, we set $\rho^* = + \infty.$ 
\end{definition}

If the rejection point $\rho^*$ is finite for some estimator, it results from the definition of the rejection point, that contamination of the estimator by measurement in the region  $\lbrace x \, | \, \IF{x} = 0 \rbrace$ does not affect the estimator in any way. If on the other hand the rejection point $\rho^*$ is not finite, it is desirable that at least
\begin{equation}
	\lim_{x \rightarrow \pm\infty} \IF{x} = 0
\end{equation}
holds.

\subsection{$M$-estimators}
$M$-estimators are generalization of maximum likelihood estimator (MLE). They are estimators defined by maximization or minimization of appropriate function $\rho(\cdot,\cdot):\mathcal{X}\times \Theta \rightarrow \mathbb{R}$. If we have parametric model $\mathcal{P} = \lbrace P_\theta : \theta \in \Theta\rbrace$, then $M$-estimator of parameter $\theta$ is defined as 
\begin{equation}
	\hat{\theta_n} = M_n(P_n) = \arg \min_{\theta \in \Theta} \sum_{i=1}^n \rho(X_i,\theta) = \arg \min_{\theta \in \Theta} \mathrm{E}_{P_n}\left[ \rho(X_i,\theta) \right],
	\label{Modhad1}
\end{equation}
where $X_1,\ldots,X_n$ is random sample with probability distribution  $P_\theta$. If the parametric model has probability density function $p_\theta$, then specific example of M-estimator is MLE which is solution of
\begin{equation}
	\hat{\theta}_{n, MLE} = \arg\min_{\theta \in \Theta} \sum_{i=1}^n \left( -\ln p_\theta(X_i)\right).
\end{equation}
If there exists
\begin{equation}
	\psi(x,\theta) = \frac{\partial}{\partial \theta} \rho(x,\theta), 
\end{equation}
then $M_n$ is the solution, or one of the solutions of equation 
\begin{equation}
	\sum_{i=1}^n \psi(X_i,\theta) = 0, \qquad \theta \in \Theta.
	\label{Modhad2}
\end{equation}
Functional relevant to $M_n$ is defined as 
\begin{equation}
	M(P) = \arg \min_{\theta \in \Theta} \mathrm{E}_{P}\left[ \rho(x,\theta) \right] = \arg \min_{\theta \in \Theta} \int_\mathcal{X} \rho(x,\theta) \, \mathrm{d}P(x),
	\label{Modhad3}
\end{equation}
or as a solution to equation 
\begin{equation}
\mathrm{E}_{P}\left[ \psi(x,\theta) \right] =  \int_\mathcal{X} \psi(x,\theta) \, \mathrm{d}P(x) = 0, \qquad M(P) = \theta \in \Theta.
\label{Modhad4}
\end{equation}
For the estimator to be unambiguous, it is required for the equation \ref{Modhad4} or for the issue  \ref{Modhad3} to have a single solution.

Here we present theorem, which allows us to derive influence function for common \ren estimator.
\begin{theorem} 
Let $\psi(\cdot,\theta) =  \frac{\partial}{\partial \theta} \rho(\cdot,\theta)$ exist and let it be absolute continuous with respect to $\theta$. Let equation \eqref{Modhad4} have single solution $M(P)$. Then if it exists, the influence function $\mathrm{IF}(x;M,P)$ is derived as
\begin{equation}
 \text{IF}(x;M,P) = -\left(\int_{\mathcal{X}} \dot{\psi} (y,M(P)) \, \mathrm{d}P(y)\right)^{-1} \psi(x,M(P)),
\end{equation}
where $\dot{\psi} (y,M(P)) = \left[\left(\frac{\mathrm{d}}{\mathrm{d}\theta}\right)^\mathrm{T}\psi(y,\theta)\right]_{\theta = M(P)}$.
\label{theo:IF}
\end{theorem}
Proof can be found in \cite{Demut2010}

\section{Decomposable pseudodistances}

In this section we will look into \ren pseudodistance. It will be clear from the following definition, that it is not typical distance, because we don't require symmetry nor triangle inequality. 

Let $\mathcal{P} = \lbrace P_\theta : \theta \in \Theta \subset \mathbb{R}^m \rbrace$ be set of probability distributions on measurable space $(\mathcal{X, A})$. Our estimates will be based on i.i.d. observations $\mathbf{X}_n = (X_1, \ldots ,X_n)$ with distribution $P_{0} \in \mathcal{F}$. Because we will be interested in robustness, we allow $P_{0} \not\in \mathcal{P}$. We therefore define set $\mathcal{P}^+ = \mathcal{P} \cup \lbrace P_0 \rbrace$.

By $\Pemp$ we denote the class of all empirical distribution functions, not necessarily from $\mathcal{P}.$
		
		\begin{definition}
	We say that the mapping $\mathfrak{D}:\mathcal{P}\times\mathcal{P}^+ \rightarrow \mathbb{R}$ is a pseudodistance between probability measures $P \in \mathcal{P}$ and $Q \in \mathcal{P}^+$ if it holds, that	
		\begin{equation}
			\mathfrak{D}(P_\theta,Q) \geq 0 \qquad \textit{ for every } \theta \in \Theta \: \text{ and for every } Q \in \mathcal{P}^+
		\end{equation}
		and 		
		\begin{equation}
			\mathfrak{D}(P_{\theta_1},P_{\theta_2})=0 \; \Leftrightarrow \; \theta_1=\theta_2 \qquad \text{ for all } \theta_1,\: \theta_2 \in \Theta.
		\end{equation}	
	This pseudodistance $\mathfrak{D}$ is called decomposable on $\mathcal{P}\times\mathcal{P}^+$ if there exist functionals 
		 $\mathfrak{D}^0:\mathcal{P}\rightarrow\mathbb{R}$, $ \mathfrak{D}^1:\mathcal{P}^+ \rightarrow \mathbb{R}$ and measurable mapping
		  $\rho_\theta : \mathbb{R}^d \rightarrow \mathbb{R}$, $ \theta \in \Theta$, so that for all $\theta \in \Theta$ and for all $Q \in \mathcal{P}^+$ the expectation $\int{\rho_\theta }\, \mathrm{d}Q$ exists and it holds, that
		\begin{equation}
			\mathfrak{D} (P_\theta, Q) = \mathfrak{D}^0 (P_\theta) + \mathfrak{D}^1 (Q) + \int \rho_\theta \, \mathrm{d}Q.
		\end{equation}
\end{definition}

\begin{definition}
	We say that a functional $T_\mathfrak{D}:\mathcal{Q} \rightarrow \Theta$, for $\mathcal{Q}=\mathcal{P}^+ \cup \mathcal{P}_{\text{emp}}$	defines minimum pseudodistance estimator (min $\mathfrak{D}$-estimator) if $\mathfrak{D}(P_\theta,Q)$ is a decomposable pseudodistance on $\mathcal{P}\times\mathcal{P}^+$ and parameters $T_\mathfrak{D}(Q) \in \Theta$ minimize $\mathfrak{D}^0 + \int{\rho_\theta}\,\mathrm{d}Q$ on $\Theta$ for all $Q \in \mathcal{Q}$, that means
	\begin{equation}
		T_\mathfrak{D}(Q) = \arg\min_{\theta \in \Theta} \left[ \mathfrak{D}^0(P_\theta) + \int{\rho_\theta}\,\mathrm{d}Q \right] \quad \forall Q \in \mathcal{Q}.
	\end{equation}
\end{definition}
In particular, for empirical distribution function $Q = P_n = \frac{1}{n}\sum_{i-1}^n \delta_{X_i} \in \mathcal{P}_{emp}$
\begin{equation}
	\hat{\theta}_{\mathfrak{D},n} =T_\mathfrak{D}(P_n)  = \arg\min_{\theta \in \Theta}\left[ \mathfrak{D}^0(P_\theta) + \dfrac{1}{n} \sum_{i=1}^n \rho_\theta (X_i) \right].
\end{equation}
\begin{theorem}
	Every min $\mathfrak{D}$-estimator 
	\begin{equation}
		\left[ \mathfrak{D}^0(P_\theta) + \dfrac{1}{n} \sum_{i=1}^n \rho_\theta (X_i) \right]
	\end{equation}
	is Fisher consistent in the sense that
\begin{equation}
	T_\mathfrak{D}(P_{\theta_0}) = \arg\min_{\theta \in \Theta} \mathfrak{D}(P_\theta, P_{\theta_0}) = \theta_0,\quad \forall \theta_0 \in \Theta.
\end{equation}
\end{theorem}
\begin{proof}
Consider any fixed $\theta_0 \in \Theta$. Then $\mathfrak{D}^1(P_\theta)$ is finite constant and from the definition of pseudodistance we get
\begin{align*}
T_\mathfrak{D}(P_{\theta_0}) & = \arg\min_{\theta \in \Theta} \left[ \mathfrak{D}^0(P_\theta) + \int{\rho_\theta}\,\mathrm{d}Q \right] \\ 
& = \arg\min_{\theta \in \Theta} \left[ \mathfrak{D}^0(P_\theta) +\mathfrak{D}^1(P_{\theta_0}) + \int{\rho_\theta}\,\mathrm{d}Q \right] \\
& = \arg\min_{\theta \in \Theta} \mathfrak{D}(P_\theta,P_{\theta_0}) \\
&= \theta_0.
\end{align*}
\end{proof}


\begin{theorem}
Let for some $\beta>0$ it holds that
	\begin{equation*}
			p^\beta, q^\beta,\ln{p} \in \mathrm{L}_1(Q), \quad \forall P \in \mathcal{P}, Q \in \mathcal{P^+}.
			\label{eq:betaCond}
	\end{equation*}
	Then for all $\alpha$, $0 < \alpha \leq \beta$, and for $P \in \mathcal{P}, \; Q \in \mathcal{P^+} $ the expression
	\begin{equation}
		\mathfrak{R}_\alpha (P,Q) = \dfrac{1}{1+\alpha}\ln{\left( \int{p^\alpha \,\mathrm{d}P } \right)} +
		\dfrac{1}{\alpha (1+\alpha)}\ln{\left( \int{q^\alpha \,\mathrm{d}Q } \right)} -
		\dfrac{1}{\alpha} \ln{\left( \int{p^\alpha \,\mathrm{d}Q } \right)}
		\label{eq:renDecDist}
	\end{equation}
		represents the family of pseudodistances decomposable in the sense of
	\begin{equation*}
		\mathfrak{R}_\alpha (P,Q) = \mathfrak{R}_\alpha^0 (P) + \mathfrak{R}_\alpha^1 (Q) - \dfrac{1}{\alpha} \ln{\left( \int{p^\alpha \,\mathrm{d}Q } \right)},
	\end{equation*}	
	where
	\begin{equation*}
		\mathfrak{R}_\alpha^0 (P) = \dfrac{1}{1+\alpha}\ln{\left( \int{p^\alpha \,\mathrm{d}P } \right)}, \quad \mathfrak{R}_\alpha^1 (Q) = \dfrac{1}{\alpha (1+\alpha)}\ln{\left( \int{q^\alpha \,\mathrm{d}Q } \right)}.
	\end{equation*}
	Moreover, for $\alpha \searrow 0$ it holds
	\begin{equation*}
		\mathfrak{R}_0 (P,Q) = \lim_{\alpha \searrow 0} \mathfrak{R}_\alpha (P,Q) =  \int{\left( \ln{q} - \ln{p} \right)\,\mathrm{d}Q}.
	\end{equation*}
\end{theorem}
Proof can be found in \cite{Decomposable2011}.

If we use \ref{eq:renDecDist} for distributions $Q \in \mathcal{P}^+ \cup \Pemp$ we get minimum \ren distance estimator defined by

\begin{equation}
	T_{\mathfrak{R}_\alpha}(Q) =
	\begin{cases}
		 \arg \min_{\theta} \left[\frac{1}{1+\alpha} \ln(\int p_\theta^\alpha\, \mathrm{d}P_\theta) - \frac{1}{\alpha} \ln(\int p_\theta^\alpha\, \mathrm{d}Q) \right] & \text{if } 0 < \alpha \leq \beta, \\
		 \arg \min_{\theta} \left[- \ln(\int p_\theta\, \mathrm{d}Q) \right] & \text{if } \alpha = 0,
	\end{cases}	
\end{equation}
which is equivalent to 
\begin{equation}
	T_{\mathfrak{R}_\alpha}(Q) = 
	\begin{cases}
		 \displaystyle{ \arg \max_{\theta \in \Theta} \left[\ln\frac{\int p_\theta^\alpha\, \mathrm{d}Q}{(\int p_\theta^\alpha\, \mathrm{d}P_\theta)^{\frac{\alpha}{1+\alpha}}} \right] }& \text{ if } 0 < \alpha \leq \beta, \\[5mm]
		 \displaystyle{ \arg \max_{\theta \in \Theta} \left[\ln(\int p_\theta\, \mathrm{d}Q) \right] }& \text{ if } \alpha = 0.
	\end{cases}	
\end{equation}

If we introduce
\begin{equation}
C_{\alpha}(\theta) = \biggl(\int p_{\theta}^{1+\alpha} \, \mathrm{d}\lambda\biggr)^{\frac{\alpha}{1+\alpha}} = \left( \int p_\theta^\alpha\, \mathrm{d}P_\theta \right)^{\frac{\alpha}{1+\alpha}},
\end{equation}
then we can write \ren estimator for $0<\alpha \leq \beta$ in the form of $M$-estimator
\begin{equation}
T_{\alpha}(Q) = \text{argmax}_{\theta} M_{\alpha}(Q,\theta), \qquad \text{where } \quad M_{\alpha}(Q,\theta) = \frac{\int p_{\theta}^{\alpha}\, \mathrm{d}Q}{C_{\alpha}(\theta)}.
\end{equation}

We are interested in estimators, in which we replace the hypothetical distribution $P_{\theta_0}$ in $\mathfrak{R}_\alpha(P_\theta, P_{\theta_0})$ by empirical distribution $P_n \in \Pemp$. In this case we can write minimum \ren distance estimator in the form

% It means that the family of minimum \ren pseudo-distance estimators defined as $\hat{\theta}_{n,\alpha} = T_{\mathfrak{R}_\alpha}(P_n)$ for $T_{\mathfrak{R}_\alpha}(Q) \in \Theta$ with $Q \in \mathcal{P}^+$ satisfies the condition

\begin{equation}
	\hat{\theta}_{\mathfrak{R}_\alpha,n} =
	\begin{cases}
		\displaystyle{ \arg \max_{\theta \in \Theta} C_\alpha\left( \theta \right)^{-1} \frac{1}{n} \sum_{i=1}^n p_{\theta}^{\alpha}\left( X_i \right) } & \text{if } 0 < \alpha \leq \beta, \\
		\displaystyle{ \arg \max_{\theta \in \Theta}  \frac{1}{n} \sum_{i=1}^n \ln p_{\theta}\left( X_i \right) } & \text{if } \alpha = 0.
	\end{cases}	
	\label{eq:renEstimator}
\end{equation}
		
In \cite{Vajda2009} author derives form of influence function for minimum \ren pseudodistance estimator according to theorem \ref{theo:IF}. If we use notation 
\begin{center}
	\begin{tabular}{c c}
	$s_\theta = \dfrac{\mathrm{d}}{\mathrm{d}\theta} \ln p_\theta, \quad$ & $ \dot{s}_\theta = \left( \dfrac{\mathrm{d}}{\mathrm{d}\theta} \right)^T s_\theta,$ \\ 
	&\\
	$c_\alpha(\theta) = \dfrac{\int p_\theta^{1+\alpha}s_\theta \mathrm{d}\lambda}{\int p_\theta^{1+\alpha} \mathrm{d}\lambda}, \quad$ & $\dot{c}_\alpha(\theta)= \left( \dfrac{\mathrm{d}}{\mathrm{d}\theta} \right)^T c_\alpha(\theta),$  \\ 
	\end{tabular} 
\end{center}
then the influence function can be written as
\begin{equation}
	\mathrm{IF}(x;T_{\mathfrak{R}_\alpha},\theta) = -\mathbf{I}^{-1}_{\alpha}(\theta) \left[ p_\theta^\alpha(x) (s_\theta (x) - c_\alpha (x)) \right], 
	\label{eq:IF}
\end{equation}
where 
\begin{equation}
\mathbf{I}_{\alpha}(\theta) = \int{ \left[\dot{s}_\theta - \dot{c}_\alpha(\theta) - \alpha(s_\theta - c_\alpha(\theta))(c^T_\alpha(\theta) - s^T_\theta) \right] p_\theta^{1+\alpha} \mathrm{d}\lambda}.
\end{equation}

\subsection{Applications for normal distribution}

Results for normal model including specific forms of estimator \eqref{eq:renEstimator} and influence function \eqref{eq:IF} were already presented in \cite{Vajda2009} and \cite{Demut2010}. We are presenting them here for completeness and because we used them in simulations in which we tested the estimators on data samples with different kind of contamination from what was used in \cite{Demut2010}. 

Minimum \ren distance estimator for $\alpha = 0$ coincides with maximum likelihood estimator
\begin{equation}
\hat{\theta}_{\mathfrak{R}_0,n} = \text{argmax}_{\theta} \frac{1}{n}\sum_{i=1}^n \ln \biggl[\frac{1}{\sqrt{2\pi \sigma^2}} \exp\biggl(-\frac{(X_i-\mu)^2}{2\sigma^2}\biggr)\biggr].
\end{equation}

Condition\eqref{eq:betaCond} holds for all $\beta > 0$, therefore for $\alpha>0$ we can transform {\mRao} \eqref{eq:renEstimator} to

\begin{equation}
	\hat{\theta}_{\mathfrak{R}_\alpha,n} = \text{argmax}_{\theta} \frac{1}{n\sigma^{\alpha/(1+\alpha)}}\sum_{i=1}^n\exp \biggl(-\alpha\frac{(X_i-\mu)^2}{2\sigma^2}\biggr).
\end{equation}

\noindent Influence function for estimator of the standard deviation $\sigma$ with the known mean $\mu = 0$ is derived in \cite{Vajda2009} and has the form

\begin{equation}
	\text{IF}(x;T_{\mathfrak{R}_{\alpha}},\sigma) = \frac{(1+\alpha)^{5/2}\sigma}{2}\biggl[\biggl(\biggl(\frac{x}{\sigma}\biggr)^2-\frac{1}{1+\alpha}\biggr) \exp\biggl(-\frac{\alpha x^2}{2\sigma^2}\biggr)\biggr].
\end{equation}
For the estimator of mean $\mu$ with fixed $\sigma$ the influence function takes the form
\begin{equation}
	\text{IF}(x;T_{\mathfrak{R}_{\alpha}},\mu) = (1+\alpha )^{3/2} (x-\mu ) \exp\left[{-\alpha\frac{(x-\mu )^2}{2 \sigma ^2}}\right].
\end{equation}
Both functions are bounded for all possible $\alpha$, which means that the estimator is B-robust. Although we do not have finite rejection point $\rho^*$, we can see, that for $\alpha>0$ the limit 
\begin{equation}
	\lim_{x \rightarrow \pm\infty} \mathrm{IF}(x;T_{\mathfrak{R}_\alpha},\cdot) = 0
\end{equation}	
holds for influence functions for both parameters.	
	
	\section{Application to specific families}
		

		\subsection{Laplace}
		
		\subsection{Exponential}
		
		\subsection{Cauchy}
		
		\subsection{Weibull}
	
\chapter{Divergence Trees}

\chapter{Goodness of fit tests}