%%%%%%%%%%%%%%%%%%%%%%%%%%%%%%%%%%%%%%%%%%%%%%%%%%%%%%%%%%%%%%%%%%%%%%%%%%%%%%%%
%%%%%%%%%%%%%%%%%%
%% 1. STRANA (DESKY)

\thispagestyle{empty}

\begin{center}

{\Large \v{C}ESK\'E VYSOK\'E U\v{C}EN\'I TECHNICK\'E V PRAZE} \\[3.5mm]
{\Large Fakulta jadern\'a a fyzik\'alně in\v{z}en\'yrsk\'a}

\vspace{\stretch{1}}

{\Huge\textbf{DIPLOMOV\'A PR\'ACE}}

\vspace{\stretch{1}}

{\Large \hspace*{1cm} 2014 \hfill Jan Ku\v{c}era\hspace*{1cm}}

\end{center}

%%%%%%%%%%%%%%%%%%%%%%%%%%%%%%%%%%%%%%%%%%%%%%%%%%%%%%%%%%%%%%%%%%%%%%%%%%%%%%%%
%%%%%%%%%%%%%%%%%%
%% TITULN\'i STRANA PR\'aCE

\newpage

\thispagestyle{empty}

\begin{center}

{\Large \v{C}ESK\'E VYSOK\'E U\v{C}EN\'I TECHNICK\'E V PRAZE} \\[3.5mm]
{\Large Fakulta jadern\'a a fyzik\'alně in\v{z}en\'yrsk\'a} \\[3.5mm]
{\Large Katedra matematiky}

\vspace{\stretch{0.75}}

{\Large DIPLOMOV\'A PR\'ACE}

\vspace{\stretch{0.5}}

{\LARGE
\textbf{V\'yvoj a testov\'an\'i nov\'ych statistick\'ych technik pro separaci sign\'al\r{u} z rozpadov\'ych proces\r{u} ve Fermilabu}
\par}

\vspace{1cm}

{\LARGE
\textbf{Development and testing of new statistical techniques for signal separation of decay processes at Fermilab}
\par}

\vspace{\stretch{1.25}}

\end{center}

\begin{tabular}{ll} 
{\Large Poslucha\v{c}:} & {\Large Bc. Jan Ku\v{c}era} \\[1mm]
{\Large \v{S}kolitel:} & {\Large Ing. V\'aclav K\r{u}s, PhD.} \\[2mm]
{\Large Akademick\'y rok:}     & {\Large 2013/2014}
\end{tabular}

%%%%%%%%%%%%%%%%%%%%%%%%%%%%%%%%%%%%%%%%%%%%%%%%%%%%%%%%%%%%%%%%%%%%%%%%%%%%%%%%
%%%%%%%%%%%%%%%%%%
%% ZAD\'aN\'i PR\'aCE

%\newpage
%
%\thispagestyle{empty}
%
%\noindent
%{\Large
%Na toto m\'isto p\v{r}ijde sv\'azat \textbf{zad\'an\'i diplomov\'e pr\'ace}!\\
%V jednom z v\'ytisk\r{u} mus\'i b\'yt \textbf{origin\'al} zad\'an\'i, v ostatn\'ich kopie.\par}

%%%%%%%%%%%%%%%%%%%%%%%%%%%%%%%%%%%%%%%%%%%%%%%%%%%%%%%%%%%%%%%%%%%%%%%%%%%%%%%%
%%%%%%%%%%%%%%%%%%
%% Podekovani

\newpage

\thispagestyle{empty}

\vspace*{\stretch{1}}

\noindent{\bf Acknowledgments}

\vspace{1.5cm}


I would first like to thank my supervisor, V\'avlav K\r{u}s who provided me with helpful suggestions and comments. 

I would also like to thank Ji\v{r}\'i Franc for many useful insights into the complex structure of FNAL data.

Last but not least, I would like to thank  Petr Vok\'a\v{c}, who has been my IT guide and therefore big time saver. 

\vspace{2.5cm}


%%%%%%%%%%%%%%%%%%%%%%%%%%%%%%%%%%%%%%%%%%%%%%%%%%%%%%%%%%%%%%%%%%%%%%%%%%%%%%%%
%%%%%%%%%%%%%%%%%%
%% \v{c}ESTN\'e PROHL\'aŠEN\'i

%\newpage
%\thispagestyle{empty}
%\vspace*{\stretch{1}}
%
%\noindent{\bf \v{c}estn\'e prohl\'ašen\'i}
%
%\vspace{0.5cm}
%
%Prohla\v{s}uji na tomto m\'ist\v{e}, \v{z}e jsem p\v{r}edlo\v{z}enou pr\'aci vypracoval samostatn\v{e} 
%a \v{z}e jsem uvedl ve\v{s}kerou pou\v{z}itou literaturu.
%
%\vspace{1.5cm}
%
%\noindent
%\begin{minipage}[b]{5cm}
%V Praze dne 3.kv\v{e}tna, 2014
%\end{minipage}
%\hfill
%\begin{minipage}[t]{5cm}
%\begin{center}
%\dotfill\\
%Jan Ku\v{c}era
%\end{center}
%\end{minipage}
%
%\vspace*{2cm}

%%%%%%%%%%%%%%%%%%%%%%%%%%%%%%%%%%%%%%%%%%%%%%%%%%%%%%%%%%%%%%%%%%%%%%%%%%%%%%%%
%%%%%%%%%%%%%%%%%%
%% CZ/EN ABSTRAKTY A KL\'i\v{c}OV\'a SLOVA

\newpage

\thispagestyle{empty}

{
\setlength{\parindent}{0pt}
\vspace*{-0.3cm}
\textit{N\'azev pr\'ace:}
\textbf{V\'yvoj a testov\'an\'i nov\'ych statistick\'ych technik pro separaci sign\'al\r{u} z rozpadov\'ych proces\r{u} ve Fermilabu} \\

\textit{Autor:} Bc. Jan Ku\v{c}era \\

\textit{Obor:} In\v{z}en\'yrsk\'a informatika \\

\textit{Zamě\v{r}en\'i:}  Softwarov\'e in\v{z}en\'yrstv\'i a matematick\'a informatika \\

\textit{Druh pr\'ace:} Diplomov\'a pr\'ace \\

\textit{Vedouc\'i pr\'ace:}  Ing. V\'aclav K\r{u}s, PhD., Katedra matematiky, Fakulta jadern\'a a fyzik\'alně in\v{z}en\'yrsk\'a, \v{C}VUT v Praze\\

\textit{Konzultant:}  --- \\

\textit{Abstrakt:} 
R\'enyiho informa\v{c}n\'i vzd\'alenosti s parametrem jsou dob\v{r}e zn\'am\'ym a velmi pou\v{z}\'ivan\'ym statistick\'ym n\'astrojem d\'iky jejich robustnosti a praktick\'e proveditelnosti. P\v{r}edstavujeme MC simula\v{c}n\'i v\'ysledky pro odhady s minim\'aln\'i R\'enyiho vzd\'alenost\'i (MReD) na datov\'ych vzorc\'ich s mal\'ym po\v{c}tem pozorov\'an\'i a ukazujeme dopad hodnoty parametru $\alpha$ na robustnost. Navrhujeme heuristiku pro takov\'e MReD odhady,  kdy p\v{r}\'isn\'a minimalizace vede k delta funkc\'im. Tato pr\'ace se tak\'e zab\'yv\'a testov\'an\'im statistick\'ych hypot\'ez a zamě\v{r}uje se na anal\'yzu rozpadov\'ych kan\'al\r{u} "lepton plus jets" z běhu RunII \v{c}\'asticov\'eho urychlova\v{c}e Tevatron ve Fermilabu. Studujeme tak\'e pou\v{z}it\'i R\'enyiho divergenc\'i jako statistiky pro porovn\'av\'an\'i m\'iry shodnosti a rozd\'ilnosti dvou datov\'ych vzork\r{u} z RunII.\\

\textit{Kl\'i\v{c}ov\'a slova:}  Odhady s minim\'aln\'i vzd\'alenost\'i, robustnost, $\phi$-divergence, top kvark, testov\'an\'i statistick\'ych hypot\'ez

\vspace{1.4cm}

\textit{Title:}
\textbf{Development and testing of new statistical techniques for signal separation of decay processes at Fermilab} \\

\textit{Author:} Bc. Jan Ku\v{c}era\\
\begin{samepage}
\textit{Abstract:} 
R\'enyi information divergences are well-known and widely used in statistical inference due to their robustness and practical feasibility. MC simulation results for the Minimum
R\'enyi Distance (MReD) estimates in the case of small sample data sets are presented and the
effect of input parameter $\alpha$ to robustness is shown. Heuristic approach is proposed for such MReD estimates when the strict minimization leads to delta functions. This thesis also studies statistical hypothesis testing and focuses on analysis of the "lepton plus jets" decay channels from RunII measured at the particle accelerator Tevatron in Fermilab. We also study the usage of R\'enyi divergence as a possible statistic for comparing similarity or dissimilarity of two datasets.
 \\

{\textit{Key words:}  Minimum distance estimators, robustness, $\phi$-divergence, top quark, statistical hypothesis testing}

\end{samepage}}
%%%%%%%%%%%%%%%%%%%%%%%%%%%%%%%%%%%%%%%%%%%%%%%%%%%%%%%%%%%%%%%%%%%%%%%%%%%%%%%%
%%%%%%%%%%%%%%%%%%%%
%% Konec uvodnich stran