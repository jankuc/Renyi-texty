%%%%%%%%%%%%%%%%%%%%%%%%%%%%%%%%%%%%%%%%%%%%%%%%%%%%%%%%%%%%%%%%%%%%%%%%%%%%%%%%
%%%%%%%%%%%%%%%%%%
%% 1. STRANA (DESKY)

\thispagestyle{empty}

\begin{center}

{\Large ČESKÉ VYSOKÉ UČENÍ TECHNICKÉ V PRAZE} \\[3.5mm]
{\Large Fakulta jaderná a fyzikálně inženýrská}

\vspace{\stretch{1}}

{\Huge\textbf{DIPLOMOVÁ PRÁCE}}

\vspace{\stretch{1}}

{\Large \hspace*{1cm} 2008 \hfill Jméno Příjmení \hspace*{1cm}}

\end{center}

%%%%%%%%%%%%%%%%%%%%%%%%%%%%%%%%%%%%%%%%%%%%%%%%%%%%%%%%%%%%%%%%%%%%%%%%%%%%%%%%
%%%%%%%%%%%%%%%%%%
%% TITULNÍ STRANA PRÁCE

\newpage

\thispagestyle{empty}

\begin{center}

{\Large ČESKÉ VYSOKÉ UČENÍ TECHNICKÉ V PRAZE} \\[3.5mm]
{\Large Fakulta jaderná a fyzikálně inženýrská} \\[3.5mm]
{\Large Katedra matematiky}

\vspace{\stretch{0.75}}

{\Large DIPLOMOVÁ PRÁCE}

\vspace{\stretch{0.5}}

{\LARGE
\textbf{Vývoj a testování nových statistických technik pro separaci signálů z rozpadových procesů ve Fermilabu}
\par}

\vspace{1cm}

{\LARGE
\textbf{Development and testing of new statistical techniques for signal separation of decay processes at Fermilab}
\par}

\vspace{\stretch{1.25}}

\end{center}

\begin{tabular}{ll} 
{\Large Posluchač:} & {\Large Bc. Jan Kučera} \\[1mm]
{\Large \v{S}kolitel:} & {\Large Ing. Václav Kůs, PhD.} \\[2mm]
{\Large Akademický rok:}     & {\Large 2013/2014}
\end{tabular}

%%%%%%%%%%%%%%%%%%%%%%%%%%%%%%%%%%%%%%%%%%%%%%%%%%%%%%%%%%%%%%%%%%%%%%%%%%%%%%%%
%%%%%%%%%%%%%%%%%%
%% ZADÁNÍ PRÁCE

\newpage

\thispagestyle{empty}

\noindent
{\Large
Na toto místo přijde svázat \textbf{zadání diplomové práce}!\\
V jednom z výtisků musí být \textbf{originál} zadání, v ostatních kopie.\par}

%%%%%%%%%%%%%%%%%%%%%%%%%%%%%%%%%%%%%%%%%%%%%%%%%%%%%%%%%%%%%%%%%%%%%%%%%%%%%%%%
%%%%%%%%%%%%%%%%%%
%% ČESTNÉ PROHLÁŠENÍ

\newpage

\thispagestyle{empty}

\vspace*{\stretch{1}}

\noindent{\bf Čestné prohlášení}

\vspace{0.5cm}

Prohlašuji na tomto místě, že jsem předloženou práci vypracoval samostatně 
a že jsem uvedl veškerou použitou literaturu.

\vspace{1.5cm}

\noindent
\begin{minipage}[b]{5cm}
V Praze dne 3.kv\v{e}tna, 2014
\end{minipage}
\hfill
\begin{minipage}[t]{5cm}
\begin{center}
\dotfill\\
Jan Kučera
\end{center}
\end{minipage}

\vspace*{2cm}

%%%%%%%%%%%%%%%%%%%%%%%%%%%%%%%%%%%%%%%%%%%%%%%%%%%%%%%%%%%%%%%%%%%%%%%%%%%%%%%%
%%%%%%%%%%%%%%%%%%
%% CZ/EN ABSTRAKTY A KLÍČOVÁ SLOVA

\newpage

\thispagestyle{empty}

{
\setlength{\parindent}{0pt}

\textit{Název práce:}
\textbf{Vývoj a testování nových statistických technik pro separaci signálů z rozpadových procesů ve Fermilabu} \\

\textit{Autor:} Bc. Jan Kučera \\

\textit{Obor:} Inženýrská informatika \\

\textit{Zaměření:}  Softwarové inženýrství a matematická informatika \\

\textit{Druh práce:} Diplomová práce \\

\textit{Vedoucí práce:}  Ing. Václav Kůs, PhD., Katedra matematiky, Fakulta jaderná a fyzikálně inženýrská, ČVUT v Praze\\

\textit{Konzultant:}  --- \\

\textit{Abstrakt:} 
Rényiho informační vzdálenosti jsou dobře známým a velmi používaným statistickým nástrojem kvůli jejich robustnosti a praktické proveditelnosti. Představujeme MC simulační výsledky pro odhady s minimální Rényiho vzdáleností (MReV) na datových vzorcích s malým počtem pozorování a ukazujeme dopad hodnoty $\alpha$ na robustnost. Navrhujeme heuristiku pro takové MReV odhady,  kdy přísná minimalizace vede k delta funkcím. Tato práce se také zabývá testováním statistických hypotéz a zaměřuje se na analýzu kanálů lepton plus jets z běhu RunII částicového urychlovače Tevatron ve Fermilabu. Studujeme také použití Rényiho divergencí jako statistiky pro porovnávání míry shodnosti a rozdílnosti dvou datových vzorků.\\

\textit{Klíčová slova:}  Odhady s minimální vzdáleností, robustnost, $\phi$-divergence, top kvark, testování statistických hypotéz

\vspace{1.4cm}

\textit{Title:}
\textbf{Development and testing of new statistical techniques for signal separation of decay processes at Fermilab} \\

\textit{Author:} Bc. Jan Kučera\\

\textit{Abstract:} 
Rényi information divergences are well-known and widely used in statistical inference due to their robustness and practical feasibility. MC simulation results for the Minimum
Rényi Distance (MReD) estimates in the case of small sample data sets are presented and the
effect of input parameter $\alpha$ to robustness is shown. Heuristic approach is proposed for such MReD estimates when the strict minimization leads to delta functions. This thesis also studies statistical hypothesis testing and focuses on analysis of the lepton plus jets channels from RunII of the particle accelerator Tevatron in Fermilab. We also study the use of Rényi divergence as a possible statistic for comparing similarity or difference of two datasets.
 \\

\textit{Key words:}  Minimum distance estimators, robustness, $\phi$-divergence, top quark, statistical hypothesis testing
}

%%%%%%%%%%%%%%%%%%%%%%%%%%%%%%%%%%%%%%%%%%%%%%%%%%%%%%%%%%%%%%%%%%%%%%%%%%%%%%%%
%%%%%%%%%%%%%%%%%%%%
%% Konec uvodnich stran