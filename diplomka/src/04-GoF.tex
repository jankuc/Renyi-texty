\chapter{Goodness of Fit Tests}
\label{ch:GoF}

Vyber promennych \newline
testy zalozene na divergencnich vzdalenostech \\
poradi promennych zalozene na div. vzdalenostech\\
 MC vs data\\
signal vs background\\

Goal of this chapter is having number of tools for measuring similarity of two data samples and for measuring their difference. There are number of two-sample goodness of fit  tests, e.g. Kolmogorov-Smirnov test, Kruskal-Wallis test and many others. 

We need two-sample goodness of fit  tests in which we can use WECDF or incorporate the data weights by some other means.
Goodness of fit tests 
In this section we consider two i.i.d. weighted random samples $\mathbf{X}^F = (X^F_1,\ldots, X^F_m)$ and $\mathbf{X}^G = (X^G_1, \ldots, X^G_n)$ with respective weights $\mathbf{w}^F = (w^F_1,\ldots, w^F_m)$ and $\mathbf{w}^G = (w^G_1,\ldots, w^G_n)$ and respective distribution functions $F$ and $G$. From these two samples, we will construct ECDFs $F_m(x), G_n(x).$ 

We will also deal with joint sample 
\begin{equation}
\mathbf{X}^{F\cup G} = (X^{F\cup G}_1, \ldots, X^{F\cup G}_{m+n}) = (X^{F}_1, \ldots, X^{F}_{m},X^G_1,\ldots, X^G_n)
\end{equation}
with weights 
\begin{equation}
\mathbf{w}^{F\cup G} = (w^{F\cup G}_1,\ldots, w^{F\cup G}_{m+n}) = (w^{F}_1,\ldots, w^{F}_{m},w^{G}_1,\ldots, w^{G}_{n}).
\end{equation}
and with its ECDF $H_{m+n}(x)$.

First we are going to list  tests and ranks which we used for analysing the data.

We are going to test the hypothesis $H_0: F = G$ against the alternative $H_1: F \neq G$. 
\section{Kolmogorov-Smirnov Test}
Kolmogorov-Smirnov goodness of fit test is based on Glivenko-Cantelli theorem \ref{theo:glivenko-cantelli}. It uses statistic $D_{m,n}$ defined as Kolmogorov distances between ECDFs of analysed samples
\begin{equation}
D_{m,n} = \sup_{x \in \mathbb{R}} |F_m(x) - G_n(x)|.
\end{equation}
If $H_0$ is true, then according to \ref{theo:glivenko-cantelli} it holds that $D_{m,n} \rightarrow 0$ almost surely for $m \rightarrow \infty$ and $n \rightarrow \infty$. Now we present theorem which can serve as a basis for exact test.

\begin{theorem}[Smirnov]
	Let us denote  
	\begin{equation}
	K(\lambda) = 1 - 2\sum_{k=1}^{\infty} (-1)^{k+1} \exp\left[ -2k^2\lambda^2\right].
	\end{equation}
	Then 
	\begin{equation}
	\lim_{m,n \rightarrow \infty} \mathrm{P}(\sqrt{\frac{mn}{m+n}}D_{m,n} < \lambda) = K(\lambda)
	\end{equation}
	holds for all $\lambda$.
\end{theorem}
\noindent Proof can be found in \cite{Smirnov1944}. 
However, in FNAL data samples every event is weighted and to to incorporate these weights into the tests we have to use WECDF in the definition of KS statistic
\begin{equation}
D_{m,n} = \sup_{x \in \mathbb{R}} |F^w_m(x) - G^w_n(x)|.
\end{equation}

\section{Cram\'{e}r-von Mises}
Two sample Cram\'{e}r-von Mises goodness of fit test is based on the $\omega^2$ criterion
\begin{equation}
\omega^2 = \frac{mn}{m+n} \int_{-\infty}^\infty \left[F_m(x) - G_n(x) \right]^2 \,\mathrm{d} H_{m+n}(x).
\end{equation} 
Statistic derived from this integral described in \cite{Anderson62} is
\begin{align}
T & = \frac{mn}{(m+n)^2}\left( \sum_{i=1}^m \left( F_m(X^F_i) - G_n(X^F_i)\right)^2 + \sum_{j=1}^n \left( F_m(X^G_j) - G_n(X^G_j)\right)^2 \right) \\
& = \frac{mn}{(m+n)^2} \sum_{i=1}^{m+n} \left( F_m(X^{F\cup G}_i) - G_n(X^{F\cup G}_i)\right)^2  %\\
%& = \frac{mn}{m+n} \sum_{i=1}^{m+n} \frac{1}{m+n} \left( F_m(X^{F\cup G}_i) - G_n(X^{F\cup %G}_i)\right)^2. 
\end{align}
%Here 1/(m+n) in the sum is weight equal for all data and it is the amount added to the ECDF at each data point. Because we don't have equally weighted data, we have to replace the fraction by weight ${w_i^{F\cup G}}$ specific for each data point. We can write the weighted statistic as
%\begin{equation}
%T^w = \frac{mn}{(m+n)} \sum_{i=1}^{m+n} w_i^{F\cup G} \left( F_m(X^{F\cup G}_i) - G_n(X^{F\cup G}_i)\right)^2. 
%\end{equation}
expected value
\begin{equation}
\mathrm{E} [T] = \frac{1}{6} \frac{1}{6(m+n)}
\end{equation}
variance 
\begin{equation}
\mathrm{Var} [T] = \frac{1}{45} \cdot \frac{m+n+1}{(m+n)^2} \cdot \frac{4mn(m+n) - 3(m^2 + n^2)-2mn}{4mn}
\end{equation}

We adjust the observed $T$ as
\begin{equation}
T_\mathrm{lim} = \frac{T-\mathrm{E}[T]}{\sqrt{45\mathrm{Var}[T]}} + \frac{1}{6}
\end{equation}
Two-sample Cram\'er-von Mises goodness of fit test can be found in \cite{Anderson62}, where it was examined in greater depth.

\section{\ren Divergence Based Ranks}
\ren divergence has proved in many fields of information theory as a very robust distance. 

Even though there haven't been devised any specific test to go with \ren divergence, we used it to rank the variables, so we had another measure to confirm our results. Moreover, we wanted not only to test similarity of MC and data, we also wanted to now the difference between signal and background channels and there are not tests for that purpose in the literature.

%\ren divergence is a $\phi-$divergence 
%\begin{equation}
%\mathfrak{D}_\phi (P,Q) = \int q \,\phi\left( \frac{p}{q} \right)\,\mathrm{d}\mu
%\end{equation}
%with the choice 
%\begin{equation}
%\phi =  ,
%\end{equation}
%so the final form is 
%\begin{equation}
%\mathfrak{R}_\alpha = \frac{1}{a(a-1)} \int p^\alpha q^{1-\alpha}
%\end{equation}

