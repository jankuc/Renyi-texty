\chapter{Goodness of fit tests}
\label{ch:GoF}

Vyber promennych \\
testy zalozene na divergencnich vzdalenostech \\
poradi promennych zalozene na div. vzdalenostech\\

We need two sample goodness of fit  tests in which we can use WECDF or incorporate the data weights by some other means.

In this section we consider two i.i.d. weighted random samples $\mathbf{X}^F = (X^F_1,\ldots, X^F_m)$ and $\mathbf{X}^G = (X^G_1, \ldots, X^G_n)$ with respective weights $\mathbf{w}^F = (w^F_1,\ldots, w^F_m)$ and $\mathbf{w}^G = (w^G_1,\ldots, w^G_n)$ and respective distribution functions $F$ and $G$.   

First we are going to list  tests and ranks which we used for analysing the data.

We are going to test the hypothesis $H_0: F = G$ against the alternative $H_1: F \neq G$. 
\section{Kolmogorov-Smirnov test}
Kolmogorov-Smirnov goodness of fit test is based on Glivenko-Cantelli theorem \ref{theo:glivenko-cantelli}. It uses statistic $D_{m,n}$ defined as Kolmogorov distances between ECDFs of analysed samples
\begin{equation}
D_{m,n} = \sup_{x \in \mathbb{R}} |F_m(x) - G_n(x)|.
\end{equation}
If $H_0$ is true, then according to \ref{theo:glivenko-cantelli} it holds that $D_{m,n} \rightarrow 0$ almost surely for $m \rightarrow \infty$ and $n \rightarrow \infty$. Now we present theorem which can serve as a basis for exact test.

\begin{theorem}[Smirnov]
	Let us denote $M = \frac{mn}{m+n}$ and let 
	\begin{equation}
	K(\lambda) = 1 - 2\sum_{k=1}^{\infty} (-1)^{k+1} \exp\left[ -2k^2\lambda^2\right].
	\end{equation}
	Then 
	\begin{equation}
	\lim_{m,n \rightarrow \infty} \mathrm{P}(\sqrt{M}D_{m,n} < \lambda) = K(\lambda)
	\end{equation}
	holds for all $\lambda$.
\end{theorem}
\noindent Proof can be found in \cite{Smirnov1944}. 
However, in FNAL data samples every event is weighted and to to incorporate these weights into the tests we have to use WECDF in the definition of KS statistic
\begin{equation}
D_{m,n} = \sup_{x \in \mathbb{R}} |F^w_m(x) - G^w_n(x)|.
\end{equation}

\section{Cram\'{e}r-von Mises}
Two-sample Cram\'er-von Mises goodness of fit test can be found in \cite{Anderson62}, there it is examined in greater depth. Because we need weighted version of the test, the statistic described here will be slightly modified. First let us introduce joint sample $\mathbf{X}^{F\cup G} = (X^{F\cup G}_1, \ldots, X^{F\cup G}_{m+n})$ with weights $\mathbf{w}^{F\cup G} = (w^{F\cup G}_1,\ldots, w^{F\cup G}_{m+n}).$ Joint wECDF is defined as 

\begin{equation}
T  = \frac{mn}{(m+n)^2}\left( \sum_{i=1}^m \left( F^w_m(X^F_i) - G^w_n(X^F_i)\right)^2 + \sum_{j=1}^n \left( F^w_m(X^G_j) - G^w_n(X^G_j)\right)^2 \right)
\end{equation}

expected value
\begin{equation}
\mathrm{E} [T] = \frac{1}{6} \frac{1}{6(m+n)}
\end{equation}
variance 
\begin{equation}
\mathrm{Var} [T] = \frac{1}{45} \cdot \frac{m+n+1}{(m+n)^2} \cdot \frac{4mn(m+n) - 3(m^2 + n^2)-2mn}{4mn}
\end{equation}

We adjust the observed $T$ as
\begin{equation}
T_\mathrm{lim} = \frac{T-\mathrm{E}[T]}{\sqrt{45\mathrm{Var}[T]}} + \frac{1}{6}
\end{equation}






