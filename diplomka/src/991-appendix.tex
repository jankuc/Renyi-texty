
%\begin{appendices}
%\addcontentsline{toc}{chapter}{Appendix}
\appendix 
\chapter{Appendix}


\begin{landscape}
\begin{table}[h]\footnotesize
\caption{Electron, Goodness of fit tests between Yield sample and DATA.}
\centering
\subfloat[][Electron, 3 jets]{
\begin{tabular}{|l|r|r|r|l|r|r|r|r|}
\hline
 & \multicolumn{ 3}{c|}{\textbf{KS}} & \multicolumn{ 3}{c|}{\textbf{CM}} & \multicolumn{ 2}{c|}{\textbf{AD}} \\ \hline
\textbf{Variable} & \multicolumn{1}{l|}{\textbf{H}} & \multicolumn{1}{l|}{\textbf{p-Val}} & \multicolumn{1}{l|}{\textbf{Stat}} & \textbf{H} & \multicolumn{1}{l|}{\textbf{p-Val}} & \multicolumn{1}{l|}{\textbf{Stat}} & \multicolumn{1}{l|}{\textbf{H}} & \multicolumn{1}{l|}{\textbf{Stat}} \\ \hline
\textsf{Apla} & 1 & 0.000 & 0.025 & 1 & 0 & 2.463 & 1 & 16.089 \\ \hline
\textsf{Spher} & 1 & 0.000 & 0.036 & 1 & 0 & 5.011 & 1 & 45.806 \\ \hline
\textsf{HTL} & 1 & 0.000 & 0.021 & 1 & 0 & 1.367 & 1 & 6.233 \\ \hline
\textsf{JetMt} & 0 & 0.025 & 0.014 & 1 & 0.005 & 0.977 & 1 & 9.819 \\ \hline
\textsf{HT3} & 0 & 0.231 & 0.010 & 0 & 0.586 & 0.100 & 0 & 0.729 \\ \hline
\textsf{MEvent} & 1 & 0.001 & 0.018 & 1 & 0 & 1.820 & 1 & 14.488 \\ \hline
\textsf{MT1NL} & 0 & 0.623 & 0.007 & 0 & 0.565 & 0.104 & 0 & 1.089 \\ \hline
\textsf{M01mall} & 0 & 0.057 & 0.012 & 0 & 0.057 & 0.441 & 0 & 2.606 \\ \hline
\textsf{M0nl} & 1 & 0.000 & 0.023 & 1 & 0 & 1.895 & 1 & 11.929 \\ \hline
\textsf{M1nl} & 0 & 0.312 & 0.009 & 0 & 0.238 & 0.216 & 0 & 2.050 \\ \hline
\textsf{Mt0nl} & 1 & 0.000 & 0.024 & 1 & 0 & 2.318 & 1 & 14.809 \\ \hline
\textsf{Met} & 1 & 0.003 & 0.017 & 1 & 0.009 & 0.770 & 1 & 4.661 \\ \hline
\textsf{Mtt} & 0 & 0.087 & 0.011 & 0 & 0.097 & 0.353 & 1 & 6.632 \\ \hline
\textsf{Mva\_max} & 0 & 0.121 & 0.012 & 0 & 0.064 & 0.420 & 1 & 4.444 \\ \hline
\textsf{Wmt} & 0 & 0.057 & 0.012 & 0 & 0.044 & 0.485 & 0 & 2.968 \\ \hline
\textsf{Wpt} & 0 & 0.801 & 0.006 & 0 & 0.820 & 0.059 & 0 & 0.547 \\ \hline
\textsf{Centr} & 1 & 0.000 & 0.032 & 1 & 0 & 1.362 & 1 & 18.727 \\ \hline
\textsf{DRminejet} & 0 & 0.030 & 0.013 & 0 & 0.013 & 0.710 & 1 & 3.894 \\ \hline
\textsf{DiJetDrmin} & 1 & 0.000 & 0.024 & 1 & 0 & 1.957 & 1 & 9.402 \\ \hline
\textsf{Ht} & 1 & 0.002 & 0.017 & 1 & 0.008 & 0.845 & 1 & 3.966 \\ \hline
\textsf{Ht20} & 0 & 0.055 & 0.012 & 0 & 0.224 & 0.225 & 1 & 3.293 \\ \hline
\textsf{Ktminp} & 0 & 0.028 & 0.013 & 0 & 0.014 & 0.688 & 1 & 3.516 \\ \hline
\textsf{Lepdphimet} & 0 & 0.060 & 0.012 & 0 & 0.114 & 0.327 & 0 & 2.177 \\ \hline
\textsf{Lepemv} & 1 & 0 & 0.214 & 0 & 0.728 & 0.074 & 1 & 12.305 \\ \hline
\end{tabular} %
\label{tab:e3j-tests}
} \quad 
\subfloat[][Electron, 4 jets]{
\begin{tabular}{|l|r|r|r|l|r|r|r|r|}
\hline
  & \multicolumn{ 3}{c|}{\textbf{KS}} & \multicolumn{ 3}{c|}{\textbf{CM}} & \multicolumn{ 2}{c|}{\textbf{AD}} \\ \hline
\textbf{Variable} & \multicolumn{1}{l|}{\textbf{H}} & \multicolumn{1}{l|}{\textbf{p-Val}} & \multicolumn{1}{l|}{\textbf{Stat}} & \textbf{H} & \multicolumn{1}{l|}{\textbf{p-Val}} & \multicolumn{1}{l|}{\textbf{Stat}} & \multicolumn{1}{l|}{\textbf{H}} & \multicolumn{1}{l|}{\textbf{Stat}} \\ \hline
\textsf{Apla} & 1 & 0.000 & 0.046 & 1 & 0 & 2.913 & 1 & 16.705 \\ \hline
\textsf{Spher} & 1 & 0.000 & 0.046 & 1 & 0 & 3.056 & 1 & 21.969 \\ \hline
\textsf{HTL} & 0 & 0.324 & 0.017 & 0 & 0.314 & 0.178 & 0 & 1.234 \\ \hline
\textsf{JetMt} & 1 & 0.002 & 0.034 & 0 & 0.013 & 0.704 & 1 & 5.346 \\ \hline
\textsf{HT3} & 0 & 0.024 & 0.027 & 1 & 0.005 & 0.965 & 1 & 5.680 \\ \hline
\textsf{MEvent} & 0 & 0.129 & 0.021 & 0 & 0.060 & 0.430 & 1 & 5.855 \\ \hline
\textsf{MT1NL} & 0 & 0.200 & 0.020 & 0 & 0.280 & 0.194 & 0 & 1.999 \\ \hline
\textsf{M01mall} & 0 & 0.020 & 0.028 & 0 & 0.016 & 0.671 & 1 & 5.293 \\ \hline
\textsf{M0nl} & 1 & 0.001 & 0.036 & 1 & 0.005 & 0.969 & 1 & 7.291 \\ \hline
\textsf{M1nl} & 0 & 0.810 & 0.012 & 0 & 0.849 & 0.054 & 0 & 1.679 \\ \hline
\textsf{Mt0nl} & 1 & 0.002 & 0.034 & 1 & 0.005 & 0.989 & 1 & 7.400 \\ \hline
\textsf{Met} & 0 & 0.212 & 0.019 & 0 & 0.143 & 0.291 & 0 & 1.546 \\ \hline
\textsf{Mtt} & 0 & 0.084 & 0.023 & 0 & 0.056 & 0.444 & 1 & 5.155 \\ \hline
\textsf{Mva\_max} & 1 & 0.002 & 0.036 & 0 & 0.030 & 0.550 & 0 & 3.115 \\ \hline
\textsf{Wmt} & 0 & 0.794 & 0.012 & 0 & 0.804 & 0.062 & 0 & 0.600 \\ \hline
\textsf{Wpt} & 0 & 0.381 & 0.017 & 0 & 0.619 & 0.093 & 0 & 0.553 \\ \hline
\textsf{Centr} & 1 & 0.002 & 0.034 & 0 & 0.019 & 0.635 & 1 & 5.993 \\ \hline
\textsf{DRminejet} & 0 & 0.062 & 0.024 & 0 & 0.160 & 0.275 & 0 & 1.521 \\ \hline
\textsf{DiJetDrmin} & 1 & 0.000 & 0.041 & 1 & 0 & 2.079 & 1 & 10.609 \\ \hline
\textsf{Ht} & 1 & 0.008 & 0.030 & 0 & 0.062 & 0.426 & 0 & 2.125 \\ \hline
\textsf{Ht20} & 0 & 0.065 & 0.024 & 0 & 0.040 & 0.501 & 0 & 2.621 \\ \hline
\textsf{Ktminp} & 1 & 0.001 & 0.037 & 1 & 0.000 & 1.426 & 1 & 7.913 \\ \hline
\textsf{Lepdphimet} & 0 & 0.928 & 0.010 & 0 & 0.899 & 0.046 & 0 & 0.425 \\ \hline
\textsf{Lepemv} & 1 & 1.38E-076 & 0.171 & 0 & 0.983 & 0.027 & 1 & 9.423 \\ \hline
\end{tabular}
\label{tab:e4j-tests}
}
\end{table}



\end{landscape}
%
\begin{landscape}
\begin{table}[h]\footnotesize
\caption{muon, Goodness of fit tests between Yield sample and DATA.}
\centering
\subfloat[][Muon, 3 jets]{
\begin{tabular}{|l|r|r|r|l|r|r|r|r|}
\hline
 & \multicolumn{ 3}{c|}{\textbf{KS}} & \multicolumn{ 3}{c|}{\textbf{CM}} & \multicolumn{ 2}{c|}{\textbf{AD}} \\ \hline
\textbf{Variable} & \multicolumn{1}{l|}{\textbf{H}} & \multicolumn{1}{l|}{\textbf{p-Val}} & \multicolumn{1}{l|}{\textbf{Stat}} & \textbf{H} & \multicolumn{1}{l|}{\textbf{p-Val}} & \multicolumn{1}{l|}{\textbf{Stat}} & \multicolumn{1}{l|}{\textbf{H}} & \multicolumn{1}{l|}{\textbf{Stat}} \\ \hline
\textsf{Apla} & 1 & 0.000 & 0.047 & 1 & 0 & 4.908 & 1 & 37.607 \\ \hline
\textsf{Spher} & 1 & 0.000 & 0.054 & 1 & 0 & 8.666 & 1 & 87.874 \\ \hline
\textsf{HTL} & 1 & 0.000 & 0.040 & 1 & 0.000 & 5.455 & 1 & 26.783 \\ \hline
\textsf{JetMt} & 1 & 0.000 & 0.047 & 1 & 0.000 & 10.337 & 1 & 63.211 \\ \hline
\textsf{HT3} & 1 & 0.007 & 0.018 & 0 & 0.015 & 0.685 & 1 & 3.921 \\ \hline
\textsf{MEvent} & 1 & 0.000 & 0.033 & 1 & 0.000 & 4.836 & 1 & 35.974 \\ \hline
\textsf{MT1NL} & 0 & 0.044 & 0.015 & 0 & 0.027 & 0.571 & 1 & 3.257 \\ \hline
\textsf{M01mall} & 1 & 0.000 & 0.028 & 1 & 0.000 & 2.786 & 1 & 16.218 \\ \hline
\textsf{M0nl} & 1 & 0.000 & 0.040 & 1 & 0.000 & 6.920 & 1 & 44.227 \\ \hline
\textsf{M1nl} & 1 & 0.000 & 0.024 & 1 & 0.000 & 1.217 & 1 & 6.467 \\ \hline
\textsf{Mt0nl} & 1 & 0.000 & 0.040 & 1 & 0.000 & 8.491 & 1 & 54.467 \\ \hline
\textsf{Met} & 0 & 0.294 & 0.010 & 0 & 0.398 & 0.147 & 0 & 1.321 \\ \hline
\textsf{Mtt} & 1 & 0.000 & 0.026 & 1 & 0.000 & 2.667 & 1 & 21.978 \\ \hline
\textsf{Mva\_max} & 0 & 0.033 & 0.017 & 0 & 0.019 & 0.627 & 1 & 5.581 \\ \hline
\textsf{Wmt} & 1 & 0.003 & 0.019 & 1 & 0.003 & 1.076 & 1 & 6.915 \\ \hline
\textsf{Wpt} & 0 & 0.223 & 0.011 & 0 & 0.198 & 0.243 & 0 & 1.575 \\ \hline
\textsf{Centr} & 1 & 0.000 & 0.047 & 1 & 0.000 & 2.585 & 1 & 37.522 \\ \hline
\textsf{DRminejet} & 1 & 0.000 & 0.026 & 1 & 0.000 & 3.045 & 1 & 18.621 \\ \hline
\textsf{DiJetDrmin} & 0 & 0.159 & 0.012 & 0 & 0.080 & 0.383 & 0 & 2.018 \\ \hline
\textsf{Ht} & 1 & 0.000 & 0.050 & 1 & 0.000 & 8.808 & 1 & 44.202 \\ \hline
\textsf{Ht20} & 1 & 0.003 & 0.019 & 1 & 0.006 & 0.941 & 1 & 7.086 \\ \hline
\textsf{Ktminp} & 0 & 0.050 & 0.014 & 0 & 0.081 & 0.382 & 0 & 1.841 \\ \hline
\textsf{Lepdphimet} & 1 & 0.008 & 0.018 & 0 & 0.017 & 0.659 & 1 & 4.278 \\ \hline
\end{tabular} %
\label{tab:m3j-tests}
} \quad 
\subfloat[][Muon, 4 jets]{
\begin{tabular}{|l|r|r|r|l|r|r|r|r|}
\hline
  & \multicolumn{ 3}{c|}{\textbf{KS}} & \multicolumn{ 3}{c|}{\textbf{CM}} & \multicolumn{ 2}{c|}{\textbf{AD}} \\ \hline
\textbf{Variable} & \multicolumn{1}{l|}{\textbf{H}} & \multicolumn{1}{l|}{\textbf{p-Val}} & \multicolumn{1}{l|}{\textbf{Stat}} & \textbf{H} & \multicolumn{1}{l|}{\textbf{p-Val}} & \multicolumn{1}{l|}{\textbf{Stat}} & \multicolumn{1}{l|}{\textbf{H}} & \multicolumn{1}{l|}{\textbf{Stat}} \\ \hline
\textsf{Apla} & 1 & 0.001 & 0.040 & 1 & 0.008 & 0.854 & 1 & -683.9 \\ \hline
\textsf{Spher} & 1 & 0.000 & 0.046 & 1 & 0 & 1.262 & 1 & -37490 \\ \hline
\textsf{HTL} & 0 & 0.128 & 0.025 & 0 & 0.105 & 0.340 & 0 & 0.124 \\ \hline
\textsf{JetMt} & 0 & 0.100 & 0.026 & 0 & 0.119 & 0.320 & 1 & -6.171 \\ \hline
\textsf{HT3} & 0 & 0.099 & 0.026 & 0 & 0.020 & 0.620 & 1 & -6.251 \\ \hline
\textsf{MEvent} & 0 & 0.273 & 0.021 & 0 & 0.112 & 0.330 & 1 & -3.522 \\ \hline
\textsf{MT1NL} & 0 & 0.024 & 0.031 & 0 & 0.156 & 0.279 & 0 & -0.332 \\ \hline
\textsf{M01mall} & 0 & 0.020 & 0.032 & 0 & 0.018 & 0.640 & 1 & -26.6 \\ \hline
\textsf{M0nl} & 0 & 0.019 & 0.032 & 0 & 0.044 & 0.486 & 1 & -4.784 \\ \hline
\textsf{M1nl} & 0 & 0.238 & 0.022 & 0 & 0.265 & 0.201 & 0 & -0.654 \\ \hline
\textsf{Mt0nl} & 0 & 0.192 & 0.023 & 0 & 0.172 & 0.263 & 0 & -0.159 \\ \hline
\textsf{Met} & 0 & 0.136 & 0.024 & 0 & 0.120 & 0.319 & 0 & -0.177 \\ \hline
\textsf{Mtt} & 0 & 0.139 & 0.024 & 0 & 0.105 & 0.339 & 1 & -5.523 \\ \hline
\textsf{Mva\_max} & 1 & 0.006 & 0.037 & 0 & 0.136 & 0.299 & 0 & -0.328 \\ \hline
\textsf{Wmt} & 0 & 0.211 & 0.022 & 0 & 0.261 & 0.203 & 0 & -0.753 \\ \hline
\textsf{Wpt} & 0 & 0.303 & 0.020 & 0 & 0.239 & 0.216 & 0 & 0.343 \\ \hline
\textsf{Centr} & 1 & 0.001 & 0.042 & 1 & 0.004 & 1.031 & 1 & -3714 \\ \hline
\textsf{DRminejet} & 0 & 0.130 & 0.024 & 0 & 0.162 & 0.273 & 0 & 0.121 \\ \hline
\textsf{DiJetDrmin} & 0 & 0.437 & 0.018 & 0 & 0.420 & 0.141 & 0 & 0.695 \\ \hline
\textsf{Ht} & 0 & 0.082 & 0.026 & 0 & 0.052 & 0.457 & 0 & -0.544 \\ \hline
\textsf{Ht20} & 1 & 0.003 & 0.038 & 1 & 0.007 & 0.901 & 1 & -42.546 \\ \hline
\textsf{Ktminp} & 0 & 0.449 & 0.018 & 0 & 0.489 & 0.122 & 0 & 0.747 \\ \hline
\textsf{Lepdphimet} & 0 & 0.652 & 0.015 & 0 & 0.566 & 0.104 & 0 & 0.693 \\ \hline
\end{tabular}
\label{tab:m4j-tests}
}
\end{table}

\end{landscape}
%
\begin{sidewaystable}[htbp] \footnotesize
\caption{GoF tests and RD statistics for electron + 2 jets.}
\centering
\begin{tabular}{|l|r|r|r|r|r|r|r|r|r|r|r|r|r|r|r|}
\hline
 & \multicolumn{3}{c|}{\textbf{KS}} & \multicolumn{3}{c|}{\textbf{CM}} & \multicolumn{2}{c|}{\textbf{AD}} & \multicolumn{6}{c|}{\textbf{\ren divergence}} \\ \hline
\textbf{Variable} & \textbf{H} & \textbf{p-Val} & \textbf{Stat} & \textbf{H} & \textbf{p-Val} & \textbf{Stat} & \textbf{H} & \textbf{Stat} & \textbf{Sqrt} & \textbf{Rice} & \textbf{Sturge} & \textbf{Doane} & \textbf{Scott} & \textbf{Kernel} \\ \hline
\textsf{Apla} & 1 & 0.001 & 0.040 & 1 & 0.008 & 0.854 & 1 & 7.833 & 0.018 & 0.014 & 0.008 & 0.010 & 0.015 & 0.012 \\ \hline
\textsf{Spher} & 1 & 0.0001 & 0.046 & 1 & 0 & 1.262 & 1 & 11.42 & 0.013 & 0.012 & 0.011 & 0.009 & 0.011 & 0.012 \\ \hline
\textsf{HTL} & 0 & 0.128 & 0.025 & 0 & 0.105 & 0.340 & 0 & 1.844 & 0.010 & 0.006 & 0.003 & 0.003 & 0.007 & 0.007 \\ \hline
\textsf{JetMt} & 0 & 0.100 & 0.026 & 0 & 0.119 & 0.320 & 1 & 3.748 & 0.016 & 0.009 & 0.007 & 0.007 & 0.008 & 0.013 \\ \hline
\textsf{HT3} & 0 & 0.099 & 0.026 & 0 & 0.020 & 0.620 & 1 & 3.758 & 0.013 & 0.005 & 0.003 & 0.004 & 0.005 & 0.008 \\ \hline
\textsf{MEvent} & 0 & 0.273 & 0.021 & 0 & 0.112 & 0.330 & 1 & 3.333 & 0.017 & 0.011 & 0.006 & 0.008 & 0.013 & 0.012 \\ \hline
\textsf{MT1NL} & 0 & 0.024 & 0.031 & 0 & 0.156 & 0.279 & 0 & 2.227 & 0.014 & 0.009 & 0.004 & 0.010 & 0.011 & 0.012 \\ \hline
\textsf{M01mall} & 0 & 0.020 & 0.032 & 0 & 0.018 & 0.640 & 1 & 4.959 & 0.013 & 0.009 & 0.005 & 0.006 & 0.016 & 0.007 \\ \hline
\textsf{M0nl} & 0 & 0.019 & 0.032 & 0 & 0.044 & 0.486 & 1 & 3.554 & 0.019 & 0.009 & 0.005 & 0.008 & 0.010 & 0.014 \\ \hline
\textsf{M1nl} & 0 & 0.238 & 0.022 & 0 & 0.265 & 0.201 & 0 & 2.424 & 0.014 & 0.008 & 0.003 & 0.005 & 0.008 & 0.014 \\ \hline
\textsf{Mt0nl} & 0 & 0.192 & 0.023 & 0 & 0.172 & 0.263 & 0 & 2.100 & 0.013 & 0.009 & 0.005 & 0.006 & 0.011 & 0.012 \\ \hline
\textsf{Met} & 0 & 0.136 & 0.024 & 0 & 0.120 & 0.319 & 0 & 2.114 & 0.009 & 0.005 & 0.003 & 0.004 & 0.005 & 0.008 \\ \hline
\textsf{Mtt} & 0 & 0.139 & 0.024 & 0 & 0.105 & 0.339 & 1 & 3.662 & 0.013 & 0.006 & 0.004 & 0.005 & 0.006 & 0.011 \\ \hline
\textsf{Mva\_max} & 1 & 0.006 & 0.037 & 0 & 0.136 & 0.299 & 0 & 2.224 & 0.019 & 0.010 & 0.005 & 0.008 & 0.003 & 0.104 \\ \hline
\textsf{Wmt} & 0 & 0.211 & 0.022 & 0 & 0.261 & 0.203 & 0 & 2.476 & 0.012 & 0.007 & 0.003 & 0.003 & 0.005 & 0.010 \\ \hline
\textsf{Wpt} & 0 & 0.303 & 0.020 & 0 & 0.239 & 0.216 & 0 & 1.579 & 0.014 & 0.007 & 0.004 & 0.004 & 0.007 & 0.005 \\ \hline
\textsf{Centr} & 1 & 0.0006 & 0.042 & 1 & 0.004 & 1.031 & 1 & 9.347 & 0.013 & 0.008 & 0.005 & 0.005 & 0.006 & 0.015 \\ \hline
\textsf{DRminejet} & 0 & 0.130 & 0.024 & 0 & 0.162 & 0.273 & 0 & 1.847 & 0.015 & 0.006 & 0.003 & 0.005 & 0.007 & 0.007 \\ \hline
\textsf{DiJetDrmin} & 0 & 0.437 & 0.018 & 0 & 0.420 & 0.141 & 0 & 0.862 & 0.008 & 0.005 & 0.002 & 0.002 & 0.005 & 0.005 \\ \hline
\textsf{Ht} & 0 & 0.082 & 0.026 & 0 & 0.052 & 0.457 & 0 & 2.361 & 0.014 & 0.008 & 0.005 & 0.006 & 0.010 & 0.008 \\ \hline
\textsf{Ht20} & 1 & 0.003 & 0.038 & 1 & 0.007 & 0.901 & 1 & 5.366 & 0.015 & 0.010 & 0.006 & 0.005 & 0.011 & 0.013 \\ \hline
\textsf{Ktminp} & 0 & 0.449 & 0.018 & 0 & 0.489 & 0.122 & 0 & 0.684 & 0.010 & 0.005 & 0.003 & 0.003 & 0.013 & 0.007 \\ \hline
\textsf{Lepdphimet} & 0 & 0.652 & 0.015 & 0 & 0.566 & 0.104 & 0 & 0.868 & 0.011 & 0.008 & 0.004 & 0.002 & 0.002 & 0.004 \\ \hline
\end{tabular}
\label{tab:muoYD4jetStats}
\end{sidewaystable}

%% KS CM AD
\begin{figure}[t]
    \centering
    \subfloat[][Electron, 2 jets]{\includegraphics[width=0.45\textwidth]{rankFigs/KS_CM_AD/KS_CM_AD-ele-data_4-njet_2-rotated90.pdf}} %
	\quad
    \subfloat[][Muon, 2 jets]{\includegraphics[width=0.45\textwidth]{rankFigs/KS_CM_AD/KS_CM_AD-muo-data_4-njet_2-rotated90.pdf}}
	\\
    \subfloat[][Electron, 3 jets]{\includegraphics[width=0.45\textwidth]{rankFigs/KS_CM_AD/KS_CM_AD-ele-data_4-njet_3-rotated90.pdf}} %
    \quad
    \subfloat[][Muon, 3 jets]{\includegraphics[width=0.45\textwidth]{rankFigs/KS_CM_AD/KS_CM_AD-muo-data_4-njet_3-rotated90.pdf}} 
    \\
    \subfloat[][Electron, 4 jets]{\includegraphics[width=0.45\textwidth]{rankFigs/KS_CM_AD/KS_CM_AD-ele-data_4-njet_4-rotated90.pdf}} %
    \quad
    \subfloat[][Muon, 4 jets]{\includegraphics[width=0.45\textwidth]{rankFigs/KS_CM_AD/KS_CM_AD-muo-data_4-njet_4-rotated90.pdf}}
\caption{Ranks based on  KS, CM, AD statistics with its median and standard deviation. Comparison between Yield sample and DATA.}
\label{fig:app-YD-KSCMAD}
\end{figure}


%% Renyi YD
\begin{figure}[t]
    \centering
    \subfloat[][Electron, 2 jets]{\includegraphics[width=0.45\textwidth]{rankFigs/renyiH/renyiH-ele-data_4-njet_2-rotated90.pdf}} %
	\quad
    \subfloat[][Muon, 2 jets]{\includegraphics[width=0.45\textwidth]{rankFigs/renyiH/renyiH-muo-data_4-njet_2-rotated90.pdf}}
	\\
    \subfloat[][Electron, 3 jets]{\includegraphics[width=0.45\textwidth]{rankFigs/renyiH/renyiH-ele-data_4-njet_3-rotated90.pdf}} %
    \quad
    \subfloat[][Muon, 3 jets]{\includegraphics[width=0.45\textwidth]{rankFigs/renyiH/renyiH-muo-data_4-njet_3-rotated90.pdf}} 
    \\
    \subfloat[][Electron, 4 jets]{\includegraphics[width=0.45\textwidth]{rankFigs/renyiH/renyiH-ele-data_4-njet_4-rotated90.pdf}} %
    \quad
    \subfloat[][Muon, 4 jets]{\includegraphics[width=0.45\textwidth]{rankFigs/renyiH/renyiH-muo-data_4-njet_4-rotated90.pdf}}
\caption{Variable ranks based on R\'enyi histograms and their median with standard deviation. Comparison between Yield sample and DATA.}
\label{fig:app-YD-RD}
\end{figure}

%% Renyi vs Renyi Kernel YD
\begin{figure}[t]
    \centering
    \subfloat[][Electron, 2 jets]{\includegraphics[width=0.45\textwidth]{rankFigs/renyiHK/renyiHK-ele-data_4-njet_2-rotated90.pdf}} %
	\quad
    \subfloat[][Muon, 2 jets]{\includegraphics[width=0.45\textwidth]{rankFigs/renyiHK/renyiHK-muo-data_4-njet_2-rotated90.pdf}}
	\\
    \subfloat[][Electron, 3 jets]{\includegraphics[width=0.45\textwidth]{rankFigs/renyiHK/renyiHK-ele-data_4-njet_3-rotated90.pdf}} %
    \quad
    \subfloat[][Muon, 3 jets]{\includegraphics[width=0.45\textwidth]{rankFigs/renyiHK/renyiHK-muo-data_4-njet_3-rotated90.pdf}} 
    \\
    \subfloat[][Electron, 4 jets]{\includegraphics[width=0.45\textwidth]{rankFigs/renyiHK/renyiHK-ele-data_4-njet_4-rotated90.pdf}} %
    \quad
    \subfloat[][Muon, 4 jets]{\includegraphics[width=0.45\textwidth]{rankFigs/renyiHK/renyiHK-muo-data_4-njet_4-rotated90.pdf}}
\caption{Median of ranks based on R\'enyi histograms with its deviation and RD based on kernel estimator. Comparison between Yield sample and DATA.}
\label{fig:app-YD-RDHK}
\end{figure}


%% MEDIANS YD
\begin{figure}[t]
    \centering
    \subfloat[][Electron, 2 jets]{\includegraphics[width=0.45\textwidth]{rankFigs/medianDiff/medianDiff-ele-data_4-njet_2-rotated90.pdf}} %
	\quad
    \subfloat[][Muon, 2 jets]{\includegraphics[width=0.45\textwidth]{rankFigs/medianDiff/medianDiff-muo-data_4-njet_2-rotated90.pdf}}
	\\
    \subfloat[][Electron, 3 jets]{\includegraphics[width=0.45\textwidth]{rankFigs/medianDiff/medianDiff-ele-data_4-njet_3-rotated90.pdf}} %
    \quad
    \subfloat[][Muon, 3 jets]{\includegraphics[width=0.45\textwidth]{rankFigs/medianDiff/medianDiff-muo-data_4-njet_3-rotated90.pdf}} 
    \\
    \subfloat[][Electron, 4 jets]{\includegraphics[width=0.45\textwidth]{rankFigs/medianDiff/medianDiff-ele-data_4-njet_4-rotated90.pdf}} %
    \quad
    \subfloat[][Muon, 4 jets]{\includegraphics[width=0.45\textwidth]{rankFigs/medianDiff/medianDiff-muo-data_4-njet_4-rotated90.pdf}}
\caption{Median of ranks based on  KS, CM, AD statistics and median based on R\'enyi histograms. Both with standard deviations. Comparison between Yield sample and DATA.}
\label{fig:app-YD-meds}
\end{figure}
%\begin{figure}[t]
    \centering
    \subfloat[][ele 3jet]{\includegraphics[width=0.45\textwidth]{rankFigs/medianDiff/medianDiff-ele-data_4-njet_2-rotated90.pdf}} %
	\quad
    \subfloat[][ele 3jet]{\includegraphics[width=0.45\textwidth]{rankFigs/medianDiff/medianDiff-muo-data_4-njet_2-rotated90.pdf}}
	\\
    \subfloat[][ele 3jet]{\includegraphics[width=0.45\textwidth]{rankFigs/medianDiff/medianDiff-ele-data_4-njet_3-rotated90.pdf}} %
    \quad
    \subfloat[][ele 3jet]{\includegraphics[width=0.45\textwidth]{rankFigs/medianDiff/medianDiff-muo-data_4-njet_3-rotated90.pdf}} 
    \\
    \subfloat[][ele 4jet]{\includegraphics[width=0.45\textwidth]{rankFigs/medianDiff/medianDiff-ele-data_4-njet_4-rotated90.pdf}} %
    \quad
    \subfloat[][ele 4jet]{\includegraphics[width=0.45\textwidth]{rankFigs/medianDiff/medianDiff-muo-data_4-njet_4-rotated90.pdf}}
\caption{cosi}
\end{figure}





%% normalized statistics YD
\begin{figure}[t]
    \centering
    \subfloat[][Electron, 2 jets]{\includegraphics[width=0.48\textwidth]{normStats-filtered/normStats-ele-data_4-njet_2-rotated90.pdf}} %
	\quad
    \subfloat[][Muon, 2 jets]{\includegraphics[width=0.48\textwidth]{normStats-filtered/normStats-muo-data_4-njet_2-rotated90.pdf}}
	\\
    \subfloat[][Electron, 3 jets]{\includegraphics[width=0.48\textwidth]{normStats-filtered/normStats-ele-data_4-njet_3-rotated90.pdf}} %
    \quad
    \subfloat[][Muon, 3 jets]{\includegraphics[width=0.48\textwidth]{normStats-filtered/normStats-muo-data_4-njet_3-rotated90.pdf}} 
    \\
    \subfloat[][Electron, 4 jets]{\includegraphics[width=0.48\textwidth]{normStats-filtered/normStats-ele-data_4-njet_4-rotated90.pdf}} %
    \quad
    \subfloat[][Muon, 4 jets]{\includegraphics[width=0.48\textwidth]{normStats-filtered/normStats-muo-data_4-njet_4-rotated90.pdf}}
\caption{Normalized statistics. Comparison between Yield sample and DATA.}
\label{fig:app-YD-stat}
\end{figure}

%% normalized sorted statistics YD
\begin{figure}[t]
    \centering
    \subfloat[][Electron, 2 jets]{\includegraphics[width=0.48\textwidth]{normStats-sorted/normStats-ele-data_4-njet_2-rotated90.pdf}%
     \label{fig:app-YD-statsSort-e2j}} %
	\quad
    \subfloat[][Muon, 2 jets]{\includegraphics[width=0.48\textwidth]{normStats-sorted/normStats-muo-data_4-njet_2-rotated90.pdf}%
     \label{fig:app-YD-statsSort-m2j}}
	\\
    \subfloat[][Electron, 3 jets]{\includegraphics[width=0.48\textwidth]{normStats-sorted/normStats-ele-data_4-njet_3-rotated90.pdf}%
     \label{fig:app-YD-statsSort-e3j}} %
    \quad
    \subfloat[][Muon, 3 jets]{\includegraphics[width=0.48\textwidth]{normStats-sorted/normStats-muo-data_4-njet_3-rotated90.pdf}%
     \label{fig:app-YD-statsSort-m3j}} 
    \\
    \subfloat[][Electron, 4 jets]{\includegraphics[width=0.48\textwidth]{normStats-sorted/normStats-ele-data_4-njet_4-rotated90.pdf}%
     \label{fig:app-YD-statsSort-e4j}} %
    \quad
    \subfloat[][Muon, 4 jets]{\includegraphics[width=0.48\textwidth]{normStats-sorted/normStats-muo-data_4-njet_4-rotated90.pdf}%
     \label{fig:app-YD-statsSort-m4j}}
\caption{Normalized statistics. Comparison between Yield sample and DATA. Variables are sorted according to RD based on doane's number of bins.}
\label{fig:app-YD-statSort}
\end{figure}


%% Medians SB
\begin{figure}[t]
    \centering
    \subfloat[][Electron, 2 jets]{\includegraphics[width=0.48\textwidth]{rankFigs/medianDiff/medianDiff-ele-data_6-njet_2-rotated90.pdf}} %
	\quad
    \subfloat[][Muon, 2 jets]{\includegraphics[width=0.48\textwidth]{rankFigs/medianDiff/medianDiff-muo-data_6-njet_2-rotated90.pdf}}
	\\
    \subfloat[][Electron, 3 jets]{\includegraphics[width=0.48\textwidth]{rankFigs/medianDiff/medianDiff-ele-data_6-njet_3-rotated90.pdf}} %
    \quad
    \subfloat[][Muon, 3 jets]{\includegraphics[width=0.48\textwidth]{rankFigs/medianDiff/medianDiff-muo-data_6-njet_3-rotated90.pdf}} 
    \\
    \subfloat[][Electron, 4 jets]{\includegraphics[width=0.48\textwidth]{rankFigs/medianDiff/medianDiff-ele-data_6-njet_4-rotated90.pdf}} %
    \quad
    \subfloat[][Muon, 4 jets]{\includegraphics[width=0.48\textwidth]{rankFigs/medianDiff/medianDiff-muo-data_6-njet_4-rotated90.pdf}}
\caption{Comparison between signal $\lbrace$\textsf{ttA\_172},  \textsf{ttAll\_172}$\rbrace$ and backgrounds. Medians of ranks based on  KS, CM, AD statistics, respective RD histograms with standard deviations.}
\label{fig:app-SB-meds}
\end{figure}




%% normalized statistics SB
\begin{figure}[t]
    \centering
    \subfloat[][Electron, 2 jets]{\includegraphics[width=0.48\textwidth]{normStats/normStats-ele-data_6-njet_2-rotated90.pdf}} %
	\quad
    \subfloat[][Muon, 2 jets]{\includegraphics[width=0.48\textwidth]{normStats/normStats-muo-data_6-njet_2-rotated90.pdf}}
	\\
    \subfloat[][Electron, 3 jets]{\includegraphics[width=0.48\textwidth]{normStats/normStats-ele-data_6-njet_3-rotated90.pdf}} %
    \quad
    \subfloat[][Muon, 3 jets]{\includegraphics[width=0.48\textwidth]{normStats/normStats-muo-data_6-njet_3-rotated90.pdf}} 
    \\
    \subfloat[][Electron, 4 jets]{\includegraphics[width=0.48\textwidth]{normStats/normStats-ele-data_6-njet_4-rotated90.pdf}} %
    \quad
    \subfloat[][Muon, 4 jets]{\includegraphics[width=0.48\textwidth]{normStats/normStats-muo-data_6-njet_4-rotated90.pdf}}
\caption{Normalized statistics. Comparison between signal $\lbrace$\textsf{ttA\_172},  \textsf{ttAll\_172}$\rbrace$ and backgrounds.}
    \label{fig:app-SB-stat}
\end{figure}

%% normalized sorted statistics SB
\begin{figure}[t]
    \centering
    \subfloat[][Electron, 2 jets]{\includegraphics[width=0.48\textwidth]{normStats-sorted/normStats-ele-data_6-njet_2-rotated90.pdf}%
     \label{fig:app-SB-statsSort-e2j}} %
	\quad
    \subfloat[][Muon, 2 jets]{\includegraphics[width=0.48\textwidth]{normStats-sorted/normStats-muo-data_6-njet_2-rotated90.pdf}%
     \label{fig:app-SB-statsSort-m2j}}
	\\
    \subfloat[][Electron, 3 jets]{\includegraphics[width=0.48\textwidth]{normStats-sorted/normStats-ele-data_6-njet_3-rotated90.pdf}%
     \label{fig:app-SB-statsSort-e3j}} %
    \quad
    \subfloat[][Muon, 3 jets]{\includegraphics[width=0.48\textwidth]{normStats-sorted/normStats-muo-data_6-njet_3-rotated90.pdf}%
     \label{fig:app-SB-statsSort-m3j}} 
    \\
    \subfloat[][Electron, 4 jets]{\includegraphics[width=0.48\textwidth]{normStats-sorted/normStats-ele-data_6-njet_4-rotated90.pdf}%
     \label{fig:app-SB-statsSort-e4j}} %
    \quad
    \subfloat[][Muon, 4 jets]{\includegraphics[width=0.48\textwidth]{normStats-sorted/normStats-muo-data_6-njet_4-rotated90.pdf}%
     \label{fig:app-SB-statsSort-m4j}}
\caption{Normalized statistics. Comparison between signal $\lbrace$\textsf{ttA\_172},  \textsf{ttAll\_172}$\rbrace$ and backgrounds. Variables are sorted according to RD based on doane's number of bins.}
    \label{fig:app-SB-statSort}
\end{figure}

\clearpage

%
%\begin{minted}[fontsize=\tiny,linenos]{c++}
%void writeCanvasFiles(TCanvas *canv, const char *fileName) {
%    char file[512];
%    sprintf(file, "%s.eps", fileName);
%    canv->Print(file);
%    sprintf(file, "%s.png", fileName);
%    canv->Print(file);
%    return;
%    gROOT->SetStyle("Plain");
%}
%
%void ControlPlotsVar(
%        char* setC, // yield
%        char* leptonC, // muo or ele
%        char* jetbinC, // 2jet or 3jet or 4jet
%        char* treeNameC, // nn_tree or 'MBC_1_1_noTrans' 
%        char* varNameC, // Apla or ttA_172
%        char* outputC, // "/work/budvar-clued0/fjfi-D0/tt_leptonjets/results_plots"
%        Int_t nBin = 25, // 25
%        Double_t intervalStart = 0.0,
%        Double_t intervalEnd = 1.0 ) {
%    
%    TString inPath = "/work/budvar-clued0/fjfi-D0/tt_leptonjets/results_root_matched_sets";
%    TString lepton = leptonC;
%    TString jetbin = jetbinC;
%    TString set = setC;
%    
%    TString treeName = treeNameC; 
%    TString varName = varNameC; //ttA_172
%    
%    TString niceName; // we will use this for printing titles and such.
%    if (treeName.CompareTo("nn_tree") == 0) { 
%    // because we are loading FLOAT variable from nn_tree and need to cast it to double
%        niceName = varName;
%    } else {
%        niceName = treeName;
%    }
%    
%    Double_t eps = 10e-5;
%    
%    const Int_t nTypes = 18;
%    TString TypePlotLeg_mu[] = {"data", "QCD", "Wlp", "Wcc", "Wbb", "ZbbMuMu", "ZbbTauTau",
%    "ZccMuMu", "ZccTauTau", "ZlpMuMu", "ZlpTauTau", "tb", "tqb", "WW", "WZ", "ZZ", "ttA_172", "ttAll_172"}; 
%    // for nice names in legends:)
%    TString TypePlotLeg_e[] = {"data", "QCD", "Wlp", "Wcc", "Wbb", "ZbbEE", "ZbbTauTau",
%    "ZccEE", "ZccTauTau", "ZlpEE", "ZlpTauTau", "tb", "tqb", "WW", "WZ", "ZZ", "ttA_172", "ttAll_172"}; 
%    // for nice names in legends:)
%    TString Type_l[nTypes];
%    TH1F* allHists[19];
%    
%    THStack *hStack = new THStack("testStackCom", Form("%s %s %s %s", niceName.Data(), lepton.Data(), jetbin.Data(), set.Data()));
%    Int_t lines[nTypes]; // just for validation of Integral()
%    Double_t weights[nTypes]; // just for validation of Integral()
%    
%    for (Int_t i = 0; i < nTypes ; i++) { // over all channels
%        
%        //  Defining Channels (muo/ele))
%        
%        if (lepton.CompareTo("muo") == 0) { // muo
%            Type_l[i] = TypePlotLeg_mu[i];
%        } else if (lepton.CompareTo("ele") == 0) { // ele
%            Type_l[i] = TypePlotLeg_e[i];
%        } else {
%            std::cout << "Wrong channel name: 'muo' or 'ele' only! You've used: '" << lepton << "'." << std::endl;
%        }
%        
%        // Loading Files
%        
%        TString filePath;
%        if (i==0) { // data (we have to load it from other directory)
%            std::cout << "Loading " << Type_l[i] << "." << std::endl;
%            filePath = Form("%s/%s/%s/%s", inPath.Data(), "data", lepton.Data(), jetbin.Data());
%        } else {
%            std::cout << "Loading " << Type_l[i] << "." << std::endl;
%            filePath = Form("%s/%s/%s/%s", inPath.Data(), set.Data(), lepton.Data(), jetbin.Data());
%        }
%        TString fileName = Form("%s/%s_miniTree.root", filePath.Data(), Type_l[i].Data());
%        TFile *f = new TFile(fileName.Data(), "READ");
%        if (f == NULL) { 
%            return;
%        }
%        f->cd();
%        
%        // Loading Trees and variables
%        // Creates histogram for current channel
%        TH1F *hist = new TH1F("h", Form("%s %s %s %s", niceName.Data(), lepton.Data(), jetbin.Data(), set.Data()), nBin, 
%        	intervalStart - eps, intervalEnd + eps);
%        Float_t weightF, valF;  // weight is Float in both nn_tree and method_trees
%        Double_t weightD, valD; // value of discriminant. Is Double_t
%        TTree *t = (TTree*) f->Get(treeName);
%        t->SetBranchStatus("*", 1);
%        if (treeName.CompareTo("nn_tree")==0) { // because we are loading FLOAT variable from nn_tree and need to cast it to double
%           t->SetBranchAddress(varName.Data(), &valF);     
%        } else { // we are loading DOUBLE variable 'ttA_172' from tree named 'method'
%           t->SetBranchAddress( varName.Data(), &valD);
%        }
%        t->SetBranchAddress("Weight", &weightF); //
%        
%        // ADDING events to HISTOGRAM
%        
%        lines[i] = 0;
%        weights[i] = 0;
%        for (Long64_t j = 0; j < t->GetEntries(); ++j) {
%            t->GetEntry(j);
%            weightD = (Double_t) weightF;
%            if (treeName.CompareTo("nn_tree") == 0) { // because we are loading float variable and need to cast it to double
%                valD = (Double_t) valF;
%            }
%            
%            // setting all datapoints to the histogram interval
%            if (valD > intervalEnd) {
%                valD = intervalEnd;
%            } else if (valD < intervalStart) {
%                valD = intervalStart;
%            } 
%            
%            hist->Fill(valD,weightD);
%            lines[i]++; // just for validating allHists[i]->Integral()
%            weights[i] += weightD; // just for validating allHists[i]->Integral()
%        }
%        allHists[i]=hist; // allHists[0] ... histogram of data.
%        allHists[i]->SetDirectory(gROOT);
%        
%        // CREATING STACK
%        
%        if (i > 0) { // We stack all but Data
%            std::cout<< "Adding " << Type_l[i] << " to stack."<<std::endl;
%            hStack->Add(((TH1F*) allHists[i]));
%        } else {
%            std::cout << "Adding " << Type_l[i] << "." << std::endl;
%        }
%        f->Close();
%    }
%    
%    
%    gROOT->SetStyle("Plain");
%    //gStyle->SetOptTitle(0); // Wouldn't print out TITLE
%    gStyle->SetOptStat(0); // 0==Doesn't print stats like MEAN, RMS.
%
%    // COLORS for the Stack.
%    setColorsForHistograms(allHists);
%    
%    // PRINT
%    
%    TCanvas *canvas = new TCanvas("canvas", "stacked hists", 0, 0, 1000, 1000);
%    canvas->Divide(1,1); // Number of subplots in canvas
%    canvas->cd(1); // cd to specific subplot 
%    allHists[0]->Draw("PE1X0"); // first plot data to have propper scale of canvas
%    hStack->Draw("same"); // plot MC in stack
%    allHists[0]->Draw("samePE1X0"); // plot data to cover the stack
%    
%    // LEGEND
%    
%    TLegend *legend = new TLegend(0.55, 0.4, 0.85, 0.9, "legend title"); //x1, y1, x2, y2
%    legend->SetTextSize(0.02);
%    legend->AddEntry(allHists[0], Form("%10.f %10i Data", allHists[0]->Integral(), lines[0] )); // data legend line
%    
%    // Stack  legend lines
%    for (Int_t i = 0; i < nTypes - 1; i++) { 
%        Int_t histLine = nTypes - 1 - i;
%        TString label = Type_l[histLine];
%        //legend->AddEntry(allHists[histLine], Form("%2.2f %2.2f   %s %1.0i",
%        TString legendLine = Form("%10.2f %10i \t %s", allHists[histLine]->Integral(), lines[histLine], label.Data());
%        legend->AddEntry(allHists[histLine], legendLine);
%    }
%    legend->Draw();
%    
%    Double_t sumW = 0;
%
%    // Difference between MC and data 
%    
%    sumW = 0;
%    for (int bin = 1; bin < nBin + 1; bin++) {
%        sumW += allHists[0]->GetArray()[bin];
%        for (int chan = 1; chan < nTypes; chan++) { // over channels
%            sumW -= allHists[chan]->GetArray()[bin];
%        }
%    }
%    std::cout << "Data - MC: " << sumW << std::endl;
%
%    // SAVING PLOTS
%    
%    TString outputFileName = Form("%s/%s/%s/%s", outputC, lepton.Data(), jetbin.Data(), niceName.Data());
%    writeCanvasFiles(canvas, outputFileName.Data());
%
%    return;
%}
%
%#ifndef __CINT__
%
%
%int main(int argc, char** argv) {
%
%    float a, b;
%    int nbin;
%    
%    if (argc == 7 ) {
%    // ControlPlotsVar yield ele 4jet nn_tree Apla ./ 25 0 0.5
%    ControlPlotsVar(argv[1], argv[2], argv[3], argv[4], argv[5], argv[6], 25, 0, 1);
%    } else if (argc == 8 ) {
%        sscanf(argv[7], "%i", &nbin);
%        ControlPlotsVar(argv[1], argv[2], argv[3], argv[4], argv[5], argv[6], nbin, 0, 1);
%    } else if (argc == 10) {
%        sscanf(argv[7], "%i", &nbin);
%        sscanf(argv[8], "%f", &a);
%        sscanf(argv[9], "%f", &b);
%        ControlPlotsVar(argv[1], argv[2], argv[3], argv[4], argv[5], argv[6], nbin, (double) a, (double) b);
%    } else {
%        std::cout << "Usage: " << argv[0] << " yield ele 4jet nn_tree Apla ./ 25 0 0.5" << std::endl;
%        exit(1);
%    }
%    return 0;
%}
%#endif
% 
%
%\end{minted}

%\end{appendices}
