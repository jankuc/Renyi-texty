\chapter{Accompanying Software}\label{ch:software}

All the software used for obtaining the results presented in this work is available for downloading or sharing via version control software \texttt{Git} at \href{https://github.com/lamatko}{\texttt{http://github.com/lamatko}}. We now briefly describe relevant projects.

Relevant projects are \href{https://github.com/lamatko/JAmde}{\texttt{JAmde}} which is a java-maven project for MC simulations of minimum distance estimators. It is able to compute tables presented in this work, compute many other divergence estimates, e.g. R\'enyi, Kolmogorov, generalized Cram\'er and Kolmogorov-Cram\'er divergences. Source code and some insight to the project was also shared with Lucie Han\'akov\'a, added some p.d.f. estimators such as stochastically equivalent histograms, kernel estimators, see \cite{Hanakova}.

\href{https://github.com/lamatko/FNAL-Gof}{\texttt{FNAL-Gof}} is the MATLAB project which comprises of many different statistical tools to analyse primarily FNAL data, but apart from database tools which are tailored specifically to FNAL data, it can be used for all feasible statistical applications. It comprises of weighted c.d.f. and p.d.f. estimators, weighted versions of many statistical tests and so on. Aforementioned database tools are already used by my colleagues working on Fermilab analyses using the MATLAB.

\href{https://github.com/lamatko/DDT}{\texttt{DDT}} is also the MATLAB project containing methods for using Divergence Decision Trees. Methods are written very generally leaving room for modifications like using different divergence criteria. It is designed for easy swapping of used divergences and classification methods. It is also used by my colleagues working on similar FNAL data analyses.

Directory \href{https://github.com/lamatko/root_scripts}{\texttt{root\_scripts}} contains primarily \texttt{C++, root, bash} scripts  and functions used for preparing the data for further use. Data obtained from FNAL in \texttt{.root} files have to undergo different processes, conversions and filtering depending on the analysis which is to follow. Some of these tasks are completed on the \texttt{.root} files because of the fast processing of large data files for which was \texttt{root} built. Final stage of the data preparations are text files, which we are able to load by any tool or programming language.
