\chapter{Fermilab data and separation strategy}

In this chapter we will present data obtained from experiment \dzero in Fermilab. 

\section{\texorpdfstring{\dzero}{D0} Experiment}
\dzero was one of experiments on synchrotron Tevatron. Main focus of the experiments there was discovery of top quark, particle predicted by Standard model and in 1995 it really was present in the measurements. Its mass is the biggest of all the subatomic particles and is subject to constant changes due to ongoing analyses. The latest value is $m_\mathrm{top} = 173.34 \pm 0.76 \,\mathrm{GeV/c}^2 $, which comes from international collaboration of CERN's experiments ATLAS, CMS on the synchrotron LHC and of Fermilab's CDF, \dzero on Tevatron \cite{jointMass}. Illustration of this result is shown in figure \ref{fig:jointMass}.

\begin{figure}[htb]
	\centering
	\includegraphics[width=0.7\textwidth]{Top-Mass-graphic-hr.jpg}
	\caption{World collaboration arrives at new best value for mass of top quark.}
	\label{fig:jointMass}
\end{figure}

\section{Top Quark}
Top quark is produced mostly in pairs top-antitop quark (written as $t\bar{t}$). In 2009 the collision was detected from which single top quark originated  (without its antiparticle). Because its high mass, top quark (and antitop quark) decays in very short time, even before forming hadrons, through weak interaction almost exclusively to $W-$boson and $b-$quark as can be seen in figure \ref{fig:feynman-ljets}. These two particles then decay in three different ways; They can form one of the leptons (electron, muon or tauon) and jets, two leptons or just jets. Figure \ref{fig:ttbarBranchingFrac} shows probability of these events. 

\begin{figure}[htb]
	\centering
	\includegraphics[width=0.5\textwidth]{top_pair_branching_frac.eps}
	\caption{Top pair branching fractions}
	\label{fig:ttbarBranchingFrac}
\end{figure}

\begin{figure}[htb]
    \centering
	\includegraphics[width=0.35\textwidth]{feynman_ttbar_ljets.png}
    \caption{Feynman diagram of top pair decay to lepton+jets: 
    $q - \text{quark}$,
    $\bar{q} - \text{antiquark}$,
	$g - \text{gluon}$,
	$t - \text{top quark}$,
    $W - \text{W--boson}$,
    $l - \text{lepton}$,
    $\nu - \text{neutrino}$,
    $b - \text{bottom quark}$ \cite{Heinson}.}
   	\label{fig:feynman-ljets}
\end{figure} 

All data are filtered many times due to number of reasons. First preprocessing takes  place in the process of collecting data events, because the data flow was simply to large to store on time. Frequency of collisions recorded by \dzero was around 1.7 MHz, which was reduced by 3 levels of triggers to 150 Hz for storing on tapes \cite{Yuntse}. Other filtering processes took place in dependence on the following analysis. One of the event filters is so called $b-$tagging -- picking those events, in which $b-$quarks are present (with some degree of certainty) as one of the results of top quark decay. This criterion ensures greater percentage of some signals in the data. According to this criterion, 3 groups are distinguished: events before tagging, events with $1\: b-$tag and $2\: b-$tags.

We have received the lepton + jets data before $b-$tagging. Specifically the Run II setup, therefore all following references will be made to that configuration. This set still has to be divided into 6 channels depending on the type of lepton and number of jets present. We therefore recognize if there is electron, or muon and also if there are $2, 3$ or more than $4$ jets. These channels have to be analysed separately due to their different physical properties.

\section{Data Dimension}
\dzero detector  produces high dimensional data (number of physical parameters  goes as high as $600$). Because a lot of the parameters are linked, badly modeled by the Monte-Carlo simulations and other reasons, there are steps to lower the dimension by transforming the data, leaving out certain variables. This process is again analysis-dependent.

Data we've obtained contain 25 variables of which not all are useful or even defined in all samples. 


dimenze\\
variables - jmena, nejake charakteristiky\\
number of dimensions\\
controlploty - pro vsechny promenne? nebo jen pro nektere?\\


