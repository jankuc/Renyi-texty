\chapter{Fermilab data and separation strategy}

In this chapter we present data which was analyzed in this work. They were obtained from the experiment \dzero in Fermilab. In the following, we talk about different data sets, because the data originate in different physical processes, some of them differ in significant physical properties and analyses which the data undertook also vary. By every mentioned set we actually mean two separate sets. The first one contains real data (or so called DATA\nomenclature{DATA}{Real measured data from experiment \dzero}) measured by the detector \dzero. The second set contains Monte Carlo simulations (so called Monte Carlo or shortly MC\nomenclature{MC}{Monte Carlo simulated data}), which is generated so that it corresponds to the measured data. It is used for training and testing of the used methods for data separation.

In order to cover all the possible effects, there are orders of magnitude more entries in the MC. Because of that, all the entries in MC are reweighted so that the volume of the weighted MC set corresponds to the volume of data. These reweighted entries (and data entries) are then called events.

\section{\texorpdfstring{\dzero}{D0} Experiment at Tevatron}
\dzero was one of the biggest experiments on Tevatron. Discovery of top quark was the main focus of the \dzero experiment on this synchrotron. Top quark was predicted by Standard model and discovered in Tevatron in 1995. Its mass is the biggest of all the subatomic particles and it is subject to constant changes due to ongoing analyses. The latest value is $m_\mathrm{top} = 173.34 \pm 0.76 \,\mathrm{GeV/c}^2 $, which comes from international collaboration of CERN's experiments ATLAS, CMS on the synchrotron LHC and of Fermilab's CDF, \dzero on Tevatron \cite{jointMass}. Illustration of this result is shown in Figure \ref{fig:jointMass}. 

\begin{figure}[h]
	\centering
	\includegraphics[width=0.6\textwidth]{Top-Mass-graphic-hr.jpg}
	\caption{World collaboration arrives at the new best value for top quark mass.}
	\label{fig:jointMass}
\end{figure}

The data samples analyzed here were collected at \dzero experiment in the run period called Run II, which started in July 2002 and ended in September 2011 \cite{Yuntse}.

\section{Top Quark} \label{sec:topQuark}
Top quark is produced mostly in pairs top-antitop quark (written as $t\bar{t}$). In 2009 the collision from which single top quark originated  (without its antiparticle) was detected. Due to its of its high mass, the top quark (and antitop quark) decays in very short time, even before forming hadrons. It happens through weak interaction top quark almost exclusively decays to $W-$boson and $b-$quark as can be seen in Figure \ref{fig:feynman-ljets}. $W-$boson then decays in three different ways: it can form one of the leptons (electron, muon or tauon) and jets, two leptons or just jets. Figure \ref{fig:ttbarBranchingFrac} shows probability of these events. 

Another parameter which changes physical properties of the collision is the number of jets present in the decay of $W-$boson. We recognize three categories: $2$ jets, $3$ jets and more than $4$ jets (shortly $4+$ jets).

\begin{figure}[thb]
  \centering
  \subfloat[t][Top pair branching fractions]{%
  	\includegraphics[width=.55\textwidth]{top_pair_branching_frac.eps}%
  	\label{fig:ttbarBranchingFrac}%
  }%
  \subfloat[t][Feynman diagram of top pair decay to lepton+jets: 
    $q - \text{quark}$,
    $\bar{q} - \text{antiquark}$,
	$g - \text{gluon}$,
	$t - \text{top quark}$,
    $W - \text{W--boson}$,
    $l - \text{lepton}$,
    $\nu - \text{neutrino}$,
    $b - \text{bottom quark}$ \cite{Heinson}]{%
  	\includegraphics[width=.4\textwidth]{feynman_ttbar_ljets.png}%
  	\label{fig:feynman-ljets}%
  }
  \caption{Decay of \ttbar pair}
\end{figure} 

All data are filtered many times due to many reasons. First preprocessing takes  place in the process of collecting data events, because the data flow was simply too large to store in time. The frequency of collisions recorded by \dzero was around 1.7 MHz, which was reduced by 3 levels of triggers to 150 Hz for storing on tapes \cite{Yuntse}. Other filtering processes were carried out in consideration to the following analysis. One of the event filters is called $b-$tagging i.e. picking those events, in which $b-$quarks are present (with specific degree of certainty) as one of the results of top quark decay. This criterion ensures greater percentage of the required signals in the dataset. According to this criterion, three groups are distinguished: events before tagging, events with one $b-$tag and two $b-$tags.

Our goal is to find separation criteria for the electron + jets and muon + jets channels before $b-$tagging. As stated above, we recognize subsets with $2, \,3,\, 4+$ jets for both channels. This means that we have six sets with slightly different physical characteristics and therefore we have to analyse them separately. 

\section{Variables}

\dzero detector  produces high dimensional data (number of physical parameters  goes as high as $600$). Because a lot of these parameters are strongly correlated, or incorrectly modeled by the Monte-Carlo simulations (in the scope of needed accuracy), there are steps to decrease the dimension by transforming the data, leaving out selected variables and so on. This process is again analysis-dependent. Data used in this work contain 24 variables of which not all are useful or even defined in all datasets. For example since \textsf{HT3} is scalar sum of the transverse momentum of the three leading jets, it is not defined for lepton + 2 jets. Furthermore \textsf{Lepemv} is output of electron discriminant, which is probability that the lepton in the decay is electron. Therefore it is not defined for the muon channel. Number of variables depending on number of jets and type of lepton are listed in table \ref{tab:data-dims}.

\begin{table}[htb]
\caption{Dimension of datasets with respect to number of jets and observed lepton.}
\centering
\begin{tabular}{|l|c|c|}
\hline 
 & electron & muon \\ 
\hline 
2 jets & 23 & 22 \\ 
\hline 
3 jets & 24 & 23 \\ 
\hline 
4+ jets & 24 & 23 \\ 
\hline 
\end{tabular} 
\label{tab:data-dims}
\end{table}

Variables are listed and briefly explained in appendix in Table \ref{tab:vars}. This selection comes from  FNAL personnel and contains these 24 variables, because they present good enough separation strength, good similarity of MC and DATA and they are in some sense complementary to the variables used by other analyses. This is required due to the possibility of combination of the analyses which benefits more from uncorrelated variables than it does from correlated. 

Former analyses were using the first 7 variables: \textsf{Apla, Spher, HTL, JetMt, HT3, MEvent, MT1NL}. One of our goals is to argue if there are variables with better separation strength or with some other better properties. 

\section{Signal and Backgrounds}
\begin{figure}[htb]
    \centering
    \subfloat[][Apla, 2 jets]{\includegraphics[width=0.30\textwidth]{Sig_vs_BG/SB_ele_njet_2_Apla.pdf}} %
	\:
    \subfloat[][Ht, 2 jets]{\includegraphics[width=0.30\textwidth]{Sig_vs_BG/SB_ele_njet_2_Ht.pdf}}
	\:
    \subfloat[][Mt0nl, 2 jets]{\includegraphics[width=0.30\textwidth]{Sig_vs_BG/SB_ele_njet_2_Mt0nl.pdf}} %
    
    \subfloat[][Apla, 3 jets]{\includegraphics[width=0.30\textwidth]{Sig_vs_BG/SB_ele_njet_3_Apla.pdf}} %
	\:
    \subfloat[][Ht, 3 jets]{\includegraphics[width=0.30\textwidth]{Sig_vs_BG/SB_ele_njet_3_Ht.pdf}}
	\:
    \subfloat[][Mt0nl, 3 jets]{\includegraphics[width=0.30\textwidth]{Sig_vs_BG/SB_ele_njet_3_Mt0nl.pdf}} %
    
    \subfloat[][Apla, 4+ jets]{\includegraphics[width=0.30\textwidth]{Sig_vs_BG/SB_ele_njet_4_Apla.pdf}} %
	\:
    \subfloat[][Ht, 4+ jets]{\includegraphics[width=0.30\textwidth]{Sig_vs_BG/SB_ele_njet_4_Ht.pdf}}
	\:
    \subfloat[][Mt0nl, 4+ jets]{\includegraphics[width=0.30\textwidth]{Sig_vs_BG/SB_ele_njet_4_Mt0nl.pdf}} %

 \caption{Histograms of signal (\textsf{ttA\_172}, \textsf{ttAll\_172}) and background channels in selected variables for electron + jets.}
\label{fig:hists-9fig}
\end{figure}

The final purpose of the analyses is to separate certain required signal from the background events comprising of other collisions, other decays and so on. Our dataset comprises of 2 signals (\textsf{ttA\_172, ttAll\_172}) and 15 backgrounds. Their list can be found in appendix in Tables \ref{tab:chnl-ele} and \ref{tab:chnl-muo}. There we can also find their proportional representation in the MC depending on the lepton and number of jets. Columns next to the percentages are number of entries (before re-weighting). Percentages in the lower part of the tables are the ratio of MC to DATA and DATA to MC respectively.

Proportional representation of the signal and backgrounds in MC can be also showed via histograms, so called control plots e.g. in the top part of Figure \ref{fig:contrPlots}. The lower part shows the similarity of MC and DATA via the ratio
\begin{equation*}
\frac{number\:of\:data\:in\:the\:bin - weights\:of\:MC\:in\:the\:bin}{number\:of\:data\:in\:the\:bin}.
\end{equation*}

\begin{figure}[thb]
  \centering
  \subfloat[t][Ht20, 3 jets, electron ]{%
  	\includegraphics[width=.65\textheight]{samples_plots-png/yield/ele/3jet/ele_3jet_Ht20.png}%
  	\label{fig:contr-ele}%
  }%
  \\
  \subfloat[t][Ht20, 3 jets, muon]{%
  	\includegraphics[width=.65\textheight]{samples_plots-png/yield/muo/3jet/muo_3jet_Ht20.png}%
  	\label{fig:contr-muo}%
  }
  \caption{So called control plots show proportional representation of signal and backgrounds in the MC and compare it to DATA.}
  \label{fig:contrPlots}
\end{figure} 

Figure \ref{fig:hists-9fig} depicts signal and background for 3 selected variables in 2, 3 and 4+ jets. We can see that the signal-to-background ratio increases with increasing number of jets. This phenomenon  can also be observed in Tables \ref{tab:chnl-ele} and \ref{tab:chnl-muo}. 
