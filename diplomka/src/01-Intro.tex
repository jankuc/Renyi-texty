\chapter*{Introduction}
\addcontentsline{toc}{chapter}{Introduction}

In this work we study statistical estimators with minimum \ren pseudodistance, mainly their robustness. In real world applications we use contaminated data quite often and some estimators, like MLE, are very inaccurate. This is motive to develop robust estimators which are immune to some kinds of contamination. Estimators based on minimization of \ren pseudodistance definitely belong to this category. In chapter \ref{ch:MDE} we derive explicit formulas for computing these estimators in various distribution families. We also propose heuristic for computation of robust minimum \ren distance estimator on small data samples, where the estimator leads to $\delta$-functions.

In chapter \ref{ch:FNAL_data} we describe data obtained from particle accelerator in Fermilab. We provide insight into the structure of the data, we briefly talk about the physical meaning.

Further, in chapter \ref{ch:GoF} we study possibilities of using statistical hypothesis tests and their use on weighted data. This is not an easy task, because the used statistic has to be able to use the weights to change the distribution of the data. Kolmogorov-Smirnov, Cramér-von Misses and Anderson-Darling tests are good candidates since their statistics use empirical c.d.f. which can be reweighted. We also study  \ren divergence as a possible statistic for comparing similarity or difference of two datasets.

We also briefly present divergence decision trees as a possible method for separation of signal.



