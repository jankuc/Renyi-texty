\chapter{\ren Divergence}
%Throughout this work we use \ren divergence in various 
Let $\mathcal{P}$ be set of probability distributions on measurable space $(\mathcal{X, A})$ and let $P,Q \in \mathcal{P} $ and let $p,q$ be their respective probability density functions. Furthermore let $\mathcal{P}$ be dominated by $\sigma$-finite measure $\lambda$.
%Let $\mathbf{X}_n = (X_1, \ldots ,X_n)$ i.i.d. observations  with distribution $P_{0} \in \mathcal{F}$. 

%\begin{definition}
%Let $\phi : (0,\infty) \rightarrow \mathbb{R}$ is convex on $(0,\infty)$, strictly convex at point point $t = 1$ and let $\phi(1) = 0$. %Let us denote $\phi(0) := \lim_{t\rightarrow 0_+} \phi(t)$ and $\phi(infty)/\infty  := \lim_{t\rightarrow +\infty} \phi(t)/t.$
%
%
%\end{definition}


\begin{definition}
Let $\Phi$ be class of all functions  $\phi(t)$, $\phi : (0,\infty) \rightarrow \mathbb{R}$ which are convex on $(0,\infty)$ and strictly convex at $t = 1$. Suppose that $\phi(1) = 0$ and $\phi(0) = \lim_{t \searrow 0} \phi(t)$ exists. Then the {\em \phidiv} measure between the probability distributions $P$ and $Q$ is defined by 
\begin{equation}
\mathfrak{D}_\phi(P,Q) = \int_\mathcal{X} q(x) \phi \left( \frac{p(x)}{q(x)}\right) \,\mathrm{d}\lambda , \qquad \phi \in \Phi.
\label{eq:phiDiv}
\end{equation}
\label{def:phiDiv}
\end{definition}
\noindent Kullback-Leibler divergence measure 
\begin{equation}
\mathfrak{D}_{KL}(P,Q) =  \int_\mathcal{X} p(x)\log\left( \frac{p(x)}{q(x)}\right) \mathrm{d}\lambda
\end{equation}
is classic example of \phidiv defined by $\phi(t) = t\log(t)$.

We will be examining even more general divergence presented by \ren in \cite{Renyi1961}. It was the first parametric generalization of $\mathfrak{D}_{KL}$, defined by
\begin{align*}
\mathfrak{R}_\alpha(P,Q) & = \frac{1}{\alpha-1} \log\int_\mathcal{X} p(x)^\alpha q(x)^{1-\alpha}\mathrm{d}\lambda , \qquad \alpha >0, \alpha \neq 1.
%& = \frac{1}{\alpha-1} \log \mathrm{E}_P\left[  \left( \frac{p(X)}{q(X)}\right)^{\alpha -1} \right] 
\end{align*}
Liese and Vajda \cite{LieseVajda1987} later extended it for all $\alpha \neq 0,1,$ by
\begin{equation}
\begin{aligned}
\mathfrak{R}_\alpha(P,Q)  = \frac{1}{\alpha(\alpha-1)} \log\int_\mathcal{X} p(x)^\alpha q(x)^{1-\alpha}\mathrm{d}\lambda.% = \frac{1}{\alpha(\alpha-1)} \log \mathrm{E}_P\left[  \left( \frac{p(X)}{q(X)}\right)^{\alpha -1} \right].
\label{eq:renDiv}
\end{aligned}
\end{equation}
In the following, expression \eqref{eq:renDiv} will be referred to as \ren divergence.
\ren divergence cannot be written in the form of $\phi-$divergence measure, it however can be written in more general form called $(h,\phi)$-divergence introduced  in \cite{Menendez1995}. It is defined by  $\mathfrak{D}_\phi^h(P,Q) = h( \mathfrak{D}_{\phi}(P,Q))$, where  $h$ is differentiable, $h(0) = 0$, $h' (0) > 0 $,  \[h : \left[0,\phi(0) + \lim_{x\rightarrow \infty} \frac{\phi(x)}{x} \right] \rightarrow [0,\infty)\] and $\phi \in \Phi$ from definition of $\phi$-divergence. For completeness we present functions $h(t), \phi(t)$ forming \ren divergence:
\begin{align*}
h(t) = & \frac{1}{\alpha(\alpha-1)} \log\left[ \alpha(\alpha - 1)t + 1\right], \qquad \alpha \neq 0,1, \\
\phi(t) = & \frac{t^\alpha - \alpha(t-1) -1}{\alpha(\alpha -1)} \qquad \alpha \neq 0,1.
\end{align*}
