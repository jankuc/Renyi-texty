\chapter*{Conclusion}
\addcontentsline{toc}{chapter}{Conclusion}

We have studied minimum \ren pseudodistance estimators and derived explicit formulas for computing these estimators in various distribution families. We have also derived formulas for their influence functions and concluded robustness of these estimators based on the properties of influence functions. We have also designed heuristic for computing minimum \ren distance estimators on small samples, which performs on these small samples better than the estimator alone. Robust simulation software was developed, which can be used further for testing other estimators. 

Aim of this work was development and testing of statistical techniques for signal separation of data obtained from \dzero experiment at particle accelerator Tevatron in Fermilab. We have studied the physical nature of the experiment and understood the restrictions it implies.

We have modified  Kolmogorov-Smirnov, Cram\'er-von Misses and Anderson-Darling tests so we could use them on weighted data samples. These tests were then used together with \ren divergence for analysing variables in \dzero data. We concluded that the use of \ren divergence as a measure for these and similar purposes will have to be modified since its results are very unstable with respect to the p.d.f. estimator used. We have proposed replacing variables \textsf{Apla} and \textsf{Spher} with variables with better separation power and better similarity of MC and DATA. We have provided software tool which can be used for further analyses on FNAL data as well as library which can be used for larger range of statistical problems.

Because of the range of this work, some tasks concerning FNAL data analysis and related problems were not discussed in detail and some were not included in this thesis.


