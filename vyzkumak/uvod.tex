\chapter*{Úvod}
 V této práci se zabýváme statistickými odhady s minimální Rényiho pseudovzdáleností a hlavně jejich robustností. Ve statistice se s odhady založenými na znečištěných datech setkáváme poměrně často a některé používané odhady jsou na těchto datech poměrně nepřesné. Proto se vyvíjí takzvané robustní odhady, kterým by určitá míra a typ znečištění neměly bránit v dobrém odhadu. Odhady založené na minimalizaci Rényiho pseudovzdáleností do této kategorie patří. V této práci odvodíme vzorce pro výpočet těchto odhadů pro různá rozdělení pravděpodobnosti a ukážeme některé jejich vlastnosti a chování. 
 
V kapitole 1 definujeme odhad s minimální vzdáleností. Kapitola 2 se pak věnuje samotné robustnosti. Zavádíme zde influenční funkci a také některé kvalitativní míry robustnosti. Dále zde definujeme M-odhad a uvedeme větu, díky které se dá spočítat influenční funkce pro obecný Rényiho odhad. V kapitole 3 definujeme odhad s minimální Rényiho pseudovzdáleností a uvedeme obecné vzorce pro jeho výpočet i pro výpočet k němu příslušné influenční funkce. V kapitole 4 pak odvodíme tvar Rényiho odhadů a jejich influenčních funkcí pro několik rodin distribucí. V 5. kapitole pak pomocí tabulek a grafů prezentujeme a porovnáváme výsledky našich experimentů používajících minimální Rényiho odhady na znečištěných datech.