\chapter{Rozložitelné pseudovzdálenosti}

V této kapitole se budeme zabývat Rényiho pseudovzdáleností. Nejde o klasickou vzdálenost, protože nepožadujeme symetrii ani splnění trojúhelníkové nerovnosti. 

Nech\v{t} $\mathcal{P} = \lbrace P_\theta : \theta \in \Theta \subset \mathbb{R}^m \rbrace$ je množina rozdělení pravděpodobností na měřitelném prostoru $\left(\mathcal{X},\mathcal{A}\right)$.
Naše odhady budou založeny na náhodném vektoru $\mathbf{X}_n = (X_1, \ldots ,X_n)$ délky $n$ s nezávislými složkami určenými stejnou distribuční funkcí $P_{r}$. Protože se budeme zabývat robustností, připouštíme $P_r \notin \mathcal{P}$. Zavádíme tedy další množinu $\mathcal{P}^+ = \mathcal{P} \cup \lbrace P_r \rbrace$. Navíc předpokládáme, že každé rozdělení $P \in \mathcal{P}^+$ je dominováno $\sigma$-finitní mírou $\lambda$ a má hustotu 
\begin{equation}
p = \dfrac{\mathrm{d}P}{\mathrm{d}\lambda}, \text{ pro každé } P \in \mathcal{P}^+.
\end{equation}
Množinu všech empirických distribucí budeme značit $\Pemp$.

\begin{definition}
	Řekneme, že zobrazení $\mathfrak{D}:\mathcal{P}\times\mathcal{P}^+ \rightarrow \mathbb{R}$ je pseudovzdálenost distribučních funkcí $P \in \mathcal{P}$ a $Q \in \mathcal{P}^+$, pokud platí		
		\begin{equation}
			\mathfrak{D}(P_\theta,Q) \geq 0 \text{ pro každé } \theta \in \Theta \text{ a pro každé } Q \in \mathcal{P}^+
		\end{equation}
		a	
		\begin{equation}
			\mathfrak{D}(P_\theta,P_{\tilde{\theta}})=0 \text{ právě když } \theta=\tilde{\theta}.
		\end{equation}	
	Řekneme, že tato pseudovzdálenost je rozložitelná, pokud existují funkcionály 
		 $\mathfrak{D}^0:\mathcal{P}\rightarrow\mathbb{R}$, $ \mathfrak{D}^1:\mathcal{P}^+ \rightarrow \mathbb{R}$ a měřitelné zobrazení 
		  $\rho_\theta : \mathcal{X} \rightarrow \mathbb{R}$, $ \theta \in \Theta$ takové, že pro každé $\theta \in \Theta$ a pro každé $Q \in \mathcal{P}^+$ existuje střední hodnota $\int{\rho_\theta }\mathrm{d}Q$ a platí
		\begin{equation}
			\mathfrak{D} (P_\theta, Q) = \mathfrak{D}^0 (P_\theta) + \mathfrak{D}^1 (Q) + \int \rho_\theta \mathrm{d}Q.
		\end{equation}
\end{definition}

\begin{definition}
	Řekneme, že funkcionál $T_\mathfrak{D}:\mathcal{Q} \rightarrow \Theta$, pro $\mathcal{Q}=\mathcal{P}^+ \cup \Pemp$	definuje odhad s minimální pseudovzdáleností (min $\mathfrak{D}$-odhad), pokud  $\mathfrak{D}(P_\theta,Q)$ je rozložitelná pseudovzdálenost na $\mathcal{P}\times\mathcal{P}^+$ a parametry  $T_\mathfrak{D}(Q) \in \Theta$ minimalizují $\mathfrak{D}^0 + \int{\rho_\theta}\mathrm{d}Q$, to znamená
	\begin{equation}
		T_\mathfrak{D}(Q) = \arg\min_{\theta \in \Theta} \left[ \mathfrak{D}^0(P_\theta) + \int{\rho_\theta}\mathrm{d}Q \right] \quad \forall Q \in \mathcal{Q}.
	\end{equation}
\end{definition}
Speciálně pro $Q = P_n = \frac{1}{n}\sum_{i-1}^n \delta_{X_i} \in \mathcal{P}_{emp}$

\begin{equation}
	\hat{\theta}_{\mathfrak{D},n} =T_\mathfrak{D}(P_n)  = \arg\min_{\theta \in \Theta}\left[ \mathfrak{D}^0(P_\theta) + \dfrac{1}{n} \sum_{i-1}^n \rho_\theta (X_i) \right].
\end{equation}
Každý min $\mathfrak{D}$-odhad je fisherovsky konzistentní, to znamená, že platí
\begin{equation}
	T_\mathfrak{D}(P_{\theta_0}) = \arg\min_{\theta \in \Theta} \mathfrak{D}(P_\theta, P_{\theta_0}) = \theta_0,\text{ pro každé }\theta_0 \in \Theta.
\end{equation}
Následující věta nám dovolí korektně definovat odhady s minimální Rényiho pseudovzdáleností.
\begin{theorem}
Nechť pro nějaké $\beta>0$ platí
	\begin{equation}
			p^\beta, q^\beta,\ln{p} \in \mathrm{L}_1(Q), \quad \forall P \in \mathcal{P}, Q \in \mathcal{P^+}.
			\label{beta-podminka}
	\end{equation}
	Potom pro každé $\alpha$ takové, že $0 < \alpha \leq \beta$, a pro $P \in \mathcal{P}, \; Q \in \mathcal{P^+} $ je 
	\begin{equation}
		\mathfrak{R}_\alpha (P,Q) = \dfrac{1}{1+\alpha}\ln{\left( \int{p^\alpha \mathrm{d}P } \right)} +
		\dfrac{1}{\alpha (1+\alpha)}\ln{\left( \int{q^\alpha \mathrm{d}Q } \right)} -
		\dfrac{1}{\alpha} \ln{\left( \int{p^\alpha \mathrm{d}Q } \right)}
	\end{equation}
	třída pseudovzdáleností rozložitelných ve smyslu 
	\begin{equation*}
		\mathfrak{R}_\alpha (P,Q) = \mathfrak{R}_\alpha^0 (P) + \mathfrak{R}_\alpha^1 (Q) - \dfrac{1}{\alpha} \ln{\left( \int{p^\alpha \mathrm{d}Q } \right)},
	\end{equation*}	
	kde 
	\begin{equation*}
		\mathfrak{R}_\alpha^0 (P) = \dfrac{1}{1+\alpha}\ln{\left( \int{p^\alpha \mathrm{d}P } \right)}, \quad \mathfrak{R}_\alpha^1 (Q) = \dfrac{1}{\alpha (1+\alpha)}\ln{\left( \int{q^\alpha \mathrm{d}Q } \right)}.
	\end{equation*}
	Navíc pro $\alpha \searrow 0$ platí
	\begin{equation*}
		\mathfrak{R}_0 (P,Q) = \lim_{\alpha \searrow 0} \mathfrak{R}_\alpha (P,Q) =  \int{\left( \ln{q} - \ln{p} \right)\mathrm{d}Q}.
	\end{equation*}
	\label{renyi-veta}
\end{theorem}
Pro $Q \in \mathcal{P}^+ \cup \Pemp$ dostáváme odhad s minimální Rényiho pseudovzdáleností ve tvaru
\begin{equation}
	T_{\mathfrak{R}_\alpha}(Q) = 
	\begin{cases}
		 \displaystyle{ \arg \min_{\theta \in \Theta} \left[\frac{1}{1+\alpha} \ln(\int p_\theta^\alpha\mathrm{d}P_\theta) - \frac{1}{\alpha} \ln(\int p_\theta^\alpha\mathrm{d}Q) \right]} & \text{ pro } 0 < \alpha \leq \beta, \\
		 \displaystyle{ \arg \min_{\theta \in \Theta} \left[- \ln(\int p_\theta\mathrm{d}Q) \right] }& \text{ pro } \alpha = 0,
	\end{cases}	
\end{equation}
tedy
\begin{equation}
	T_{\mathfrak{R}_\alpha}(Q) = 
	\begin{cases}
		 \displaystyle{ \arg \max_{\theta \in \Theta} \left[\ln\frac{\int p_\theta^\alpha\mathrm{d}Q}{(\int p_\theta^\alpha\mathrm{d}P_\theta)^{\frac{\alpha}{1+\alpha}}} \right] }& \text{ pro } 0 < \alpha \leq \beta, \\
		 \displaystyle{ \arg \max_{\theta \in \Theta} \left[\ln(\int p_\theta\mathrm{d}Q) \right] }& \text{ pro } \alpha = 0.
	\end{cases}	
\end{equation}
Pokud za $Q$ dosadíme empirické rozdělení pravděpodobnosti $P_n$ a označíme 
\begin{equation}
	C_\alpha (\theta) = \left(\int p_\theta^\alpha\mathrm{d}P_\theta \right)^{\frac{\alpha}{1+\alpha}},
\end{equation}
dostáváme odhad 
\begin{equation}
	\theta_{\alpha,n} = 
	\begin{cases}
		\displaystyle{ \arg \max_{\theta \in \Theta} C_\alpha\left( \theta \right)^{-1} \frac{1}{n} \sum_{i=1}^n p_{\theta}^{\alpha}\left( X_i \right) } & \text{ pro } 0 < \alpha \leq \beta, \\
		\displaystyle{ \arg \max_{\theta \in \Theta}  \frac{1}{n} \sum_{i=1}^n \ln p_{\theta}\left( X_i \right) } & \text{ pro } \alpha = 0.
	\end{cases}	
	\label{Renyi-estimator_formula}
\end{equation}

V \cite{Vajda2009} autor odvozuje tvar influenční funkce pro odhady s minimální Rényiho pseudovzdáleností. Pokud označíme 
\begin{center}
	\begin{tabular}{c c}
	$s_\theta = \dfrac{\mathrm{d}}{\mathrm{d}\theta} \ln p_\theta, \quad$ & $ \dot{s}_\theta = \left( \dfrac{\mathrm{d}}{\mathrm{d}\theta} \right)^T s_\theta,$ \\ 
	&\\
	$c_\alpha(\theta) = \dfrac{\int p_\theta^{1+\alpha}s_\theta \mathrm{d}\lambda}{\int p_\theta^{1+\alpha} \mathrm{d}\lambda}, \quad$ & $\dot{c}_\alpha(\theta)= \left( \dfrac{\mathrm{d}}{\mathrm{d}\theta} \right)^T c_\alpha(\theta),$  \\ 
	\end{tabular} 
\end{center}
pak má influenční funkce tvar 
\begin{equation}
	\mathrm{IF}(x;T_{\mathfrak{R}_\alpha},\theta) = -\mathbf{I}^{-1}_{\alpha}(\theta) \left[ p_\theta^\alpha(x) (s_\theta (x) - c_\alpha (x)) \right], 
	\label{IF}
\end{equation}
kde 
\begin{equation}
\mathbf{I}_{\alpha}(\theta) = \int{ \left[\dot{s}_\theta - \dot{c}_\alpha(\theta) - \alpha(s_\theta - c_\alpha(\theta))(c^T_\alpha(\theta) - s^T_\theta) \right] p_\theta^{1+\alpha} \mathrm{d}\lambda}.
\end{equation}

