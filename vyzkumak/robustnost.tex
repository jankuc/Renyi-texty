\chapter{Robustnost}

Nech\v{t} $\mathcal{P} = \lbrace P_\theta : \theta \in \Theta \subset \mathbb{R}^m \rbrace$ je množina rozdělení pravděpodobností na měřitelném prostoru $\left(\mathcal{X},\mathcal{A}\right)$. Nech\v{t} $T: \mathcal{P} \rightarrow \mathbb{R}^m$ je fisherovksy konzistentní funkcionál, to znamená, že $T(P_\theta) = \theta$ pro každé $\theta \in \Theta$. Zavedeme značení pro konvexní směs rozdělení. 

\begin{definition}
	Nech\v{t} $\varepsilon \in [0,1]$ a nech\v{t} $P, Q \in \mathcal{P}$. Pak konvexní směs těchto rozdělení značíme 
	\begin{equation}
		P_\varepsilon(Q) = (1-\varepsilon)P + \varepsilon Q.
		\label{konvex-smes}
	\end{equation}
\end{definition}

\section{Influenční funkce}
\noindent
Nyní definujeme influenční funkci, která měří vliv jednoho pozorování s hodnotou $x$ na odhad definovaný pomocí funkcionálu $T$. 

\begin{definition}
	Nech\v{t} $\delta_x$ označuje rozdělení náhodné veličiny degenerované v bodě $x,\, x \in \mathcal{X}$. Potom influenční funkce $\IF{x}$ funkcionálu $T$ v bodě $P,\, P \in \mathcal{P}$ je definována vztahem
	\begin{equation}
		\IF{x} = \lim_{\varepsilon \rightarrow 0_+} \frac{T(P_\varepsilon(\delta_x)) - T(P)}{\varepsilon} = \lim_{\varepsilon \rightarrow 0_+} \frac{T((1-\varepsilon)P + \varepsilon\delta_x) - T(P)}{\varepsilon}.
	\end{equation} 
\end{definition}

\noindent
Pokud tedy $\IF{x}$ není omezená, potom i jediné odlehlé pozorování může způsobit naprosté selhání odhadu $T$. 

Nyní zavedeme míry robustnosti, které pokrývají některá možná narušení našeho modelu. 

\begin{definition}
	Globální citlivostí funkcionálu $T$ pro rozdělení pravděpodobnosti $P$ nazýváme veličinu $\gamma^*$ definovanou jako
	\begin{equation}
		\gamma^* = \sup_{x \in \mathcal{X}} |\IF{x}|.
	\end{equation}
\end{definition}
Tato míra přibližně ukazuje nejhorší možný vliv výskytu hrubé chyby v datech na hodnotu odhadu. Z hlediska míry robustnosti je tedy žádoucí, aby hodnota $\gamma^*$ byla konečná. Odhady, pro které $\gamma*$ má konečnou hodnotu pak nazýváme B-robustní.

\begin{definition}
	Lokální citlivostí funkcionálu $T$ pro rozdělení pravděpodobnosti $P$ nazýváme veličinu $\lambda^*$ definovanou jako
	\begin{equation}
			\lambda^* = \sup_{x,y \in \mathcal{X},x \neq y}  \left| \frac{\IF{y} - \IF{x}}{y-x} \right|.
	\end{equation}
\end{definition}

\noindent Nechť je rozdělení $P$ symetrické okolo nuly. Pak můžeme definovat další míru robustnosti.

\begin{definition}
	Bod zamítání funkcionálu $T$ pro rozdělení pravděpodobnosti $P$ značíme $\rho^*$ a definujeme jako
	\begin{equation}
			\rho^* = \inf \lbrace r>0 \, | \, \IF{x} = 0 \quad pro \quad |x| > r\rbrace.
	\end{equation}
\end{definition}

\noindent Existuje-li pro nějaký odhad $\rho^*$ konečné, vyplývá z definice bodu zamítání, že kontaminace pozorováními v oblasti $\lbrace x \, | \, \IF{x} = 0 \rbrace$ nijak neovlivní výsledný odhad. Pokud u zkoumaného odhadu není $\rho^*$ konečné, je z hlediska robustnosti vůči odlehlým pozorováním vhodné, aby platilo alespoň
\begin{equation}
	\lim_{x \rightarrow \pm\infty} \IF{x} = 0.
\end{equation}

\section{$M$-odhady}
$M$-odhady jsou zobecněním maximálně věrohodného odhadu. Jsou to totiž odhady definované pomocí maximalizace, nebo minimalizace vhodně zvolené funkce $\rho(X_i,\theta):\mathcal{X}\times \Theta \rightarrow \mathbb{R}$, kde $X_1,\ldots,X_n$ je náhodný výběr s distribuční funkcí $P_\theta$ a hustotou $p_\theta$. $M$-odhad parametru $\theta$ je tedy definován, jako 
\begin{equation}
	\hat{\theta_n} = M_n(P_n) = \arg \min_{\theta \in \Theta} \sum_{i=1}^n \rho(X_i,\theta) = \arg \min_{\theta \in \Theta} \mathrm{E}_{P_n}\left[ \rho(X,\theta) \right].
	\label{Modhad1}
\end{equation}

\noindent Jestliže existuje 
\begin{equation}
	\psi(X,\theta) = \frac{\partial}{\partial \theta} \rho(X,\theta), 
\end{equation}
pak $M_n$ je řešením, nebo jedním z řešení rovnice 
\begin{equation}
	\sum_{i=1}^n \psi(X,\theta) = 0, \qquad \theta \in \Theta.
	\label{Modhad2}
\end{equation}
Funkcionál příslušný $M_n$ je definován 
\begin{equation}
	M(P) = \arg \min_{\theta \in \Theta} \mathrm{E}_{P}\left[ \rho(x,\theta) \right] = \arg \min_{\theta \in \Theta} \int_\mathcal{X} \rho(x,\theta) \mathrm{d}P(x),
	\label{Modhad3}
\end{equation}
nebo jako řešení rovnice 
\begin{equation}
\mathrm{E}_{P}\left[ \psi(x,\theta) \right] =  \int_\mathcal{X} \psi(x,\theta) \mathrm{d}P(x) = 0, \qquad M(P) = \theta \in \Theta.
\label{Modhad4}
\end{equation}
Aby byl odhad jednoznačný, je potřeba, aby tato rovnice, popřípadě úloha \eqref{Modhad3} měla jediné řešení.

A nyní již věta, která nám dá návod, jak později počítat influenční funkce pro Rényiho odhady. 
\begin{theorem} 
 Nechť existuje $\psi(\cdot,\theta) =  \frac{\partial}{\partial \theta} \rho(\cdot,\theta)$ a je absolutně spojitá vzhledem k $\theta$. Nech\v{t} rovnice \eqref{Modhad4} má jediné řešení $M(P)$ a nechť existuje influenční funkce $\mathrm{IF}(x;M,P)$. Potom platí
\begin{equation}
 \text{IF}(x;M,P) = -\left(\int_{\mathcal{X}} \dot{\psi} (y,M(P)) \mathrm{d}P(y)\right)^{-1} \psi(x,M(P)),
\end{equation}
kde $\dot{\psi} (y,M(P)) = \left[\left(\frac{\mathrm{d}}{\mathrm{d}\theta}\right)^\mathrm{T}\psi(y,\theta)\right]_{\theta = M(P)}$.
\label{influencni veta}
\end{theorem}


