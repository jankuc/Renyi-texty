 \chapter*{Závěr}
 
 V této práci jsme se zabývali odhady s minimální Rényiho pseudovzdáleností. Spočítali jsme jejich konkrétní tvar pro odhady v Laplaceově, exponenciálním, Cauchyho a Weibullově modelu. Pro tato pravděpodobnostní rozdělení jsme také odvodili tvary jim příslušných influenčních funkcí. V experimentech jsme testovali robustnost minimálních Rényiho odhadů na různě znečištěných datech. 
 
Oproti očekávání jsme zjistili, že se Rényiho odhady chovají pro různé distribuce různě. U Cauchyho rodiny změna parametru $\alpha$ mění robustnost odhadu a můžeme ho tak nastavit na právě řešený problém.  Navíc se v něm dobře odhaduje parametr měřítka, jak jsme očekávali. Poměrně univerzálním se zde zdá být parametr $\alpha = 0.2$.  U Laplaceova rozdělení odhad funguje velmi dobře i při odhadování polohy. Pro exponenciální rozdělení je sice také důležitá volba $\alpha$, ale zdá se, že příliš malé, nebo velké hodnoty nemá smysl používat a opět se jeví jako poměrně rozumná volba hodnoty okolo $\alpha = 0.2$.