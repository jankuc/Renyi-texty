\documentclass[11pt,a4paper]{report}
\usepackage[czech]{babel}
\usepackage[utf8]{inputenc}
\usepackage[T1]{fontenc}

%\frenchspacing

\usepackage[ figurename = Obr., tablename = Tab.]{caption}

%\usepackage{tikz}
%PDF
%\usepackage{lmodern}
%\usepackage{textcomp}
%\usepackage{alltt}
%\usepackage{indentfirst}
%\usepackage{color}
%\usepackage{pxfonts}

\usepackage{epsfig}

%\usepackage{index}
\usepackage{graphicx}
\usepackage{amsmath}
\usepackage{amsfonts}
\usepackage{amssymb}
%\usepackage{listings}


\setlength{\textheight}{597pt} %% cca 21cm
\setlength{\textwidth}{426pt} %% cca 15cm 
\setlength{\topmargin}{25pt}
\setlength{\oddsidemargin}{32pt}

\frenchspacing % aktivuje pou�it� n�kter�ch �esk�ch typografick�ch pravidel
% definice makra pro �esk� uvozovky:
\def\bq{\mbox{\kern.1ex\protect\raisebox{-1.3ex}[0pt][0pt]{''}\kern-.1ex}}
\def\eq{\mbox{\kern-.1ex``\kern.1ex}}
\def\ifundefined#1{\expandafter\ifx\csname#1\endcsname\relax }%
\ifundefined{uv}%
        \gdef\uv#1{\bq #1\eq}
\fi

%\usepackage[pdftex]{hyperref}

\newtheorem{theorem}{Věta}
%\newtheorem{proposition}{Proposition}
%\newtheorem {corolary}{Corolary}
\newtheorem {lemma}{Lemma}
%\newtheorem {exercise}{Exercise}
\newtheorem {definition}{Definice}
\newenvironment{proof}[1][Důkaz]{\par\noindent{\textbf{#1.}
}}{\hfill $\Box$}
%{\theorembodyfont{\upshape}  \newtheorem{remark}{Poznámka}}
%{\theorembodyfont{\upshape}  \newtheorem {example}{Příklad}}

\newcommand{\Pemp}{\mathcal{P}_\mathrm{emp}}
\newcommand{\IF}[1]{\mathrm{IF}({#1};T,P)}
\newcommand{\mRao}{mininální $\mathfrak{R}_\alpha$-odhad}
%\newcommand{\IF}{\IF{x}}