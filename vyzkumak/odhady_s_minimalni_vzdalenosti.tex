\chapter{Odhady s minimální vzdáleností}

V této kapitole budeme uvažovat parametrizovanou množinu distribučních funkcí $\mathcal{P} = \lbrace P_\theta : \theta \in \Theta \subset \mathbb{R}^m \rbrace$, indexovanou parametrem z metrického prostoru $\Theta \subset \mathbb{R}^m$ vybaveného eukleidovskou vzdáleností 
\begin{equation}
	\rho (\theta_1,\theta_2) = \sqrt{(\theta_1-\theta_2)(\theta_1-\theta_2)^T}. 
\end{equation}
Dále budeme předpokládat identifikovatelnost rodiny $\mathcal{P}$, tedy 
\begin{equation}
\theta_1 \neq \theta_2 \Rightarrow P_{\theta_1} \neq P_{\theta_2}
\end{equation}
a také měřitelnost všech $\mathcal{P}_\theta(x), x \in \mathbb{R}$. Naše odhady budou založeny na náhodném vektoru $\mathbf{X}_n = (X_1, \ldots ,X_n)$ délky $n$ s nezávislými složkami určenými stejnou distribuční funkcí $P_{\theta_0} \in \mathcal{P}$. Potom $P_{\theta_0}$ nazýváme skutečnou distribuční funkcí a $\theta_0 \in \Theta$ skutečným parametrem. 

Předpokládáme, že je rodina $\mathcal{P}$ dominována $\sigma$-finitní mírou $\lambda$. Pak označíme Radon-Nikodymovu hustotu 
\begin{equation}
p_\theta = \dfrac{\mathrm{d} P_\theta}{\mathrm{d} \lambda}.
\end{equation}

\begin{definition}
	Nech\v{t} 
\end{definition}