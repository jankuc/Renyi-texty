\chapter{Odhady s minimální vzdáleností}

V této kapitole budeme uvažovat parametrizovanou množinu distribučních funkcí $\mathcal{F} = \lbrace F_\theta : \theta \in \Theta \subset \mathbb{R}^m \rbrace$, indexovanou parametrem z metrického prostoru $\Theta \subset \mathbb{R}^m$ vybaveného eukleidovskou vzdáleností 
\begin{equation}
	\rho (\theta_1,\theta_2) = \sqrt{(\theta_1-\theta_2)(\theta_1-\theta_2)^T}. 
\end{equation}
Dále budeme předpokládat identifikovatelnost rodiny $\mathcal{F}$, tedy 
\begin{equation}
\theta_1 \neq \theta_2 \Rightarrow F_{\theta_1} \neq F_{\theta_2}
\end{equation}
a také měřitelnost všech $\mathcal{F}_\theta(x), x \in \mathbb{R}$. Naše odhady budou založeny na náhodném vektoru $\mathbf{X}_n = (X_1, \ldots ,X_n)$ délky $n$ s nezávislými složkami určenými stejnou distribuční funkcí $F_{\theta_0} \in \mathcal{F}$. Potom $F_{\theta_0}$ nazýváme skutečnou distribuční funkcí a $\theta_0 \in \Theta$ skutečným parametrem. 

Předpokládáme, že je rodina $\mathcal{F}$ dominována $\sigma$-finitní mírou $\lambda$. Pak označíme Radon-Nikodymovu hustotu 
\begin{equation}
p_\theta = \dfrac{\mathrm{d} F_\theta}{\mathrm{d} \lambda}.
\end{equation}

\begin{definition}
%	Řekneme, že posloupnost náhodných veličin $\lbrace X_n \rbrace$ s distribučními funkcemi $\lbrace F_n \rbrace$ je omezená v pravděpodobnosti, pokud pro každé $\varepsilon > 0 $ existují $M$ a $N$ takové, že 
%	\begin{equation}
%		F_n(M) - F_n(-M) > 1-\varepsilon, \quad \forall n > N.
%	\end{equation}
%	Tuto skutečnost značíme $X_n = O_p(1)$.
	Nech\v{t} $X_n$ je posloupnost náhodných veličin. Pak značíme $X_n = o_p(1)$, pokud $\lim_{n \rightarrow \infty } X_n = 0$ podle pravděpodobnosti a $X_n = o_p(\varepsilon_n)$, pokud $\frac{X_n}{\varepsilon_n} = o_p(1)$ pro nějakou posloupnost $\varepsilon_n \searrow 0$.
\end{definition}

\begin{definition}
	Nech\v{t} $\mathfrak{D}(F.G)$ je vzdálenost na $\mathcal{F}(\mathbb{R})$ a $F_n$ je nějaký odhad distribuční funkce $F_{\theta_0}$. Měřitelnou funkci $\hat{\theta}_n$ takovou, že spl\v{n}uje podmínku
	\begin{equation}
		\mathfrak{D}(F_n, F_{\hat{\theta}_n}) = \inf_{\theta \in \Theta}(F_n, F_{\theta_0}) \quad s.j.
		\label{MDE}
	\end{equation}
	nazýváme odhadem s minimální vzdáleností (MDE). Pokud je $\mathcal{F}$ dominována $\lambda$, pak $f_{\hat{\theta}_n}$  je odhad hustoty s minimální vzdáleností. Pokud místo podmínky \eqref{MDE} platí 
	\begin{equation}
		\mathfrak{D}(F_n, F_{\hat{\theta}_n}) - \inf_{\theta \in \Theta}(F_n, F_{\theta_0}) = o_p(n^{-\frac{1}{2}}), 
	\end{equation}
	říkáme, že $\hat{\theta}_n$ je přibližný odhad s minimální vzdáleností (AMDE) a příslušný  $f_{\hat{\theta}_n}$ pak přibližný odhad hustoty s minimální vzdáleností.
\end{definition}

