\documentclass[11pt, a4paper]{article}
\usepackage[utf8]{inputenc}
\usepackage[czech]{babel}
\usepackage[T1]{fontenc}

\setlength{\textheight}{597pt} %% cca 21cm
\setlength{\textwidth}{426pt} %% cca 15cm 
\setlength{\topmargin}{25pt}
\setlength{\oddsidemargin}{32pt}

\usepackage{pdflscape}
\usepackage{amsmath}

\newcommand{\intpa}{\int p_\theta^{1+\alpha}(x) \, \mathrm{d}x }
\newcommand{\fn}{\frac{1}{n} \sum_{i=1}^n p_{\theta}^{\alpha}\left( x_i \right)}
\newcommand{\fln}{\frac{1}{n} \sum_{i=1}^n \ln p_{\theta}\left( x_i \right)}
\newcommand{\Cat}{C_\alpha\left( \theta \right)}
\newcommand{\amtiT}{\arg \max_{\theta \in \Theta}}
\newcommand{\fa}{\frac{\alpha}{1+\alpha}}

\begin{document}
Označíme
\begin{equation}
	\Cat = \left( \intpa \right)^{\frac{\alpha}{1+\alpha}}
\end{equation}
pak Rényiho odhady po dosazení empirického rozdělení pravděpodobnosti $ P_n $ vypadají
\begin{equation}
	\theta_{\alpha,n} = 
	\begin{cases}
		\displaystyle{ \amtiT \Cat^{-1} \fn } & \text{pro } 0 < \alpha \leq \beta, \\
		\displaystyle{ \amtiT  \fln } & \text{pro } \alpha = 0
	\end{cases}	
\end{equation}



\section{Exponenciální rozdělení} %%%%%%%%%%%%%%%%%%%%%%%%%%%%%%%%%%%%%%     EXOPNENTIAL   %%%%%%%%%%%%%%%%%%%%%%%%%%%%%%%%%%%%%%%%%%%%%%%%%%%%%%

\begin{eqnarray}
\intpa & = & \int_{\mu }^{\infty } \left( {\frac{1}{\theta} \exp{ \left[ -\frac{(x -\mu )}{\theta } \right] }} \right) ^{1 + \alpha} \, \mathrm{d}x \nonumber\\
 & = & \int_{\mu }^{\infty } {\frac{1}{\theta^{ 1 + \alpha}} \exp{ \left[ -\frac{(1 + \alpha )(x -\mu )}{\theta } \right] }} \, \mathrm{d}x \nonumber
\end{eqnarray} 

substituujeme $ y = \frac{(1+\alpha)(x-\mu)}{\theta} $, tedy $\mathrm{d}y = \frac{1+\alpha}{\theta}\mathrm{d}x $, pak

\begin{eqnarray}
\intpa & = & \frac{1}{\theta^{ 1 + \alpha}} \int_{0}^{\infty } {\frac{\theta}{1+\alpha} \exp{ \left[ -y \right] }} \, \mathrm{d}x \nonumber\\
& = & \frac{\theta ^{-\alpha }}{1+\alpha }.
\end{eqnarray}
Následující postup nefunguje, protože v empirické distribuci stále ještě není zahrnut interval, na kterém je rozdělení definováno.
\begin{eqnarray}
	\theta_{\alpha,n} &= &\amtiT \left( \frac{\theta ^{-\alpha }}{1+\alpha } \right)^{-\frac{\alpha}{1+\alpha}} \frac{1}{n} \sum_{i=1}^n \frac{1}{\theta^\alpha} \exp \left[-\alpha\frac{x_i-\mu}{\theta} \right] \nonumber\\
	&=&\amtiT \theta^{-\frac{\alpha}{1+\alpha}} \frac{1}{n}\sum_{i=1}^n \exp \left[-\alpha\frac{x_i-\mu}{\theta} \right]
\end{eqnarray}



\section{Laplaceovo rozdělení} %%%%%%%%%%%%%%%%%%%%%%%%%%%%%%%%%%%%%%    LAPLACE    %%%%%%%%%%%%%%%%%%%%%%%%%%%%%%%%%%%%%%%%%%%%%%%%%%%%%%

\begin{eqnarray}
\intpa & = & \int_{-\infty }^{\infty } \left( {\frac{1}{2\theta} \exp{ \left[ -\frac{|x -\mu |}{\theta } \right] }} \right) ^{1 + \alpha} \, \mathrm{d}x \nonumber\\
 & = & \int_{-\infty }^{\mu } {\frac{1}{ (2\theta)^{ 1 + \alpha}} \exp{ \left[ \frac{(1 + \alpha )(x -\mu )}{\theta } \right] }} \, \mathrm{d}x \nonumber\\
 & + & \int_{\mu }^{\infty } {\frac{1}{ (2\theta)^{ 1 + \alpha}} \exp{ \left[ -\frac{(1 + \alpha )(x -\mu )}{\theta } \right] }} \, \mathrm{d}x \nonumber
\end{eqnarray} 

substituujeme $ y = \frac{(1+\alpha)(x-\mu)}{\theta} $, tedy $\mathrm{d}y = \frac{1+\alpha}{\theta}\mathrm{d}x $, pak

\begin{eqnarray}
\intpa & = & \frac{1}{ (2\theta)^{ 1 + \alpha}} \frac{\theta}{1+\alpha} \left( 
\int_{-\infty }^{0 } {\exp{ \left[ y \right] }} \, \mathrm{d}x + 
\int_{0 }^{\infty } {\exp{ \left[ -y \right] }} \, \mathrm{d}x \right) \nonumber\\
& = & \frac{1}{ (2\theta)^{ 1 + \alpha}} \frac{\theta}{1+\alpha} \cdot 2  \nonumber\\
& = & \frac{(2\theta)^{-\alpha}}{(1+\alpha)} 
\end{eqnarray}
tedy 
\begin{eqnarray}
	\theta_{\alpha,n} &= &\amtiT \left( \frac{(2\theta)^{-\alpha}}{(1+\alpha)}  \right)^{-\frac{\alpha}{1+\alpha}} \frac{1}{n} \sum_{i=1}^n \frac{1}{(2\theta)^{\alpha}} \exp \left[-\alpha\frac{|x_i-\mu|}{\theta} \right] \nonumber\\
	&=& \amtiT (2\theta)^{-\frac{\alpha}{1+\alpha}} \frac{1}{n} \sum_{i=1}^n \exp \left[-\alpha\frac{|x_i-\mu|}{\theta} \right]
\end{eqnarray}


\section{Rovnoměrné rozdělení} 

\begin{eqnarray}
\intpa & = & \int_{a }^{b } \left( \frac{1}{b-a} \right) ^{1 + \alpha} \, \mathrm{d}x \nonumber\\
% & = & \int_{a }^{b } \frac{1}{(b-a)^{(1+\alpha)}} \, \mathrm{d}x\\ 
& = & \frac{b-a}{(b-a)^{\alpha+1}} \nonumber
\end{eqnarray} 
tedy
\begin{equation}
\intpa = (b-a)^{-\alpha}
\end{equation}
Následující postup nefunguje, protože v empirické distribuci stále ještě není zahrnut interval, na kterém je rozdělení definováno. 
\begin{eqnarray}
	\theta_{\alpha,n} & = & \amtiT \left( (b-a)^{-\alpha} \right)^{-\frac{\alpha}{1+\alpha}} \frac{1}{n} \sum_{i=1}^n \frac{\mathbf{I}_{(a,b)}}{(b-a)^\alpha} \nonumber\\
	& = & \amtiT \left( (b-a)^{-\alpha} \right)^{-\frac{\alpha}{1+\alpha}} \frac{1}{n} \sum_{i=1}^n \frac{\mathbf{I}_{(a,b)}}{\left(2\sqrt{3\mathrm{V}x_i}\right)^\alpha} \nonumber\\
	& = & \amtiT (b-a)^{\frac{\alpha^2}{1+\alpha}}\frac{1}{\left(2\sqrt{3\mathrm{V}X}\right)^\alpha}	\nonumber\\
%	& = & \amtiT (b-a)^{\frac{\alpha^2 - \alpha - \alpha^2}{1+\alpha}}	\nonumber\\
%	& = & \amtiT  (b-a)^{-\frac{\alpha}{1+\alpha}}
\end{eqnarray}


\section{Cauchyovo rozdělení} %%%%%%%%%%%%%%%%%%%%%%%%%%%%%%%%%%%%%%     CAUCHY    %%%%%%%%%%%%%%%%%%%%%%%%%%%%%%%%%%%%%%%%%%%%%%%%%%%%%%

\begin{eqnarray}
\intpa & = & \int_{-\infty }^{\infty } \left( \frac{1}{\pi \sigma} \left( 1 + \left( \frac{x - \mu}{\sigma} \right) ^2 \right)^{-1} \right)^{1+\alpha} \, \mathrm{d}x \nonumber\\
% & = & \int_{a }^{b } \frac{1}{(b-a)^{(1+\alpha)}} \, \mathrm{d}x\\ 
& = & \ldots \nonumber
\end{eqnarray} 
pak
\begin{equation}
\intpa = \frac{1}{\pi^{\frac{1}{2}+\alpha}\sigma^\alpha} \frac{\Gamma(\frac{1}{2} + \alpha)}{\alpha\Gamma(\alpha)}
\end{equation}
tedy
\begin{eqnarray}
	\theta_{\alpha,n} & = & \amtiT \left( \frac{1}{\pi^{\frac{1}{2}+\alpha}\sigma^\alpha} \frac{\Gamma(\frac{1}{2} + \alpha)}{\Gamma(1+\alpha)} \right)^{-\frac{\alpha}{1+\alpha}} 
	\frac{1}{n} \sum_{i=1}^n \frac{1}{\pi^\alpha\sigma^\alpha}\left( 1 + \left( \frac{x_i-\mu}{\sigma} \right)^2 \right)^{-\alpha} \nonumber \\
	& = & \amtiT \sigma^{-\frac{\alpha}{1+\alpha}} \frac{1}{n} \sum_{i=1}^n \left( 1 + \left( \frac{x_i-\mu}{\sigma} \right)^2 \right)^{-\alpha} \nonumber
\end{eqnarray}


\section{Weibullovo rozdělení} %%%%%%%%%%%%%%%%%%%%%%%%%%%%%%%%%%%%%%    WEIBULL    %%%%%%%%%%%%%%%%%%%%%%%%%%%%%%%%%%%%%%%%%%%%%%%%%%%%%%

\begin{eqnarray}
\intpa & = & \int_{-\mu }^{\infty } \left( \frac{k}{\lambda} \left( \frac{x-\mu}{\lambda} \right)^{k-1} \exp \left[ -\left( \frac{x-\mu}{\lambda} \right)^k \right] \right)^{1+\alpha} \mathrm{d}x \nonumber\\
% & = & \int_{a }^{b } \frac{1}{(b-a)^{(1+\alpha)}} \, \mathrm{d}x\\ 
& = & \ldots \nonumber
\end{eqnarray} 
pak
\begin{equation}
\intpa = \frac{k^\alpha}{\lambda^\alpha}(1+\alpha)^{-\frac{1+\alpha +k}{k}} \Gamma \left( \frac{1+\alpha +k}{k}\right)
\end{equation}
Následující postup nefunguje, protože v empirické distribuci stále ještě není zahrnut interval, na kterém je rozdělení definováno.
\begin{eqnarray}
\theta_{\alpha,n} & = & \amtiT \left(\frac{k^\alpha}{\lambda^\alpha}(1+\alpha)^{-\frac{1+\alpha+k}{k}} \Gamma \left(\frac{1+\alpha+k}{k}\right) \right)^{-\fa} \nonumber \\
&&  \frac{1}{n} \sum_{i=1}^n \frac{k^\alpha}{\lambda^\alpha} \left( \frac{x_i-\mu}{\lambda}\right)^{\alpha(k-1)} \exp \left[-\alpha \left(\frac{x_i-\mu}{\lambda}\right)^k \right] \nonumber \\
& = & \amtiT \left( \frac{k}{\lambda} \right)^\fa (1+\alpha)^{\fa\frac{1+\alpha+k}{k}} \Gamma\left(\frac{1+\alpha+k}{k}\right)^{-\fa} \nonumber\\
&& \frac{1}{n}\sum_{i=1}^n \left( \frac{x_i-\mu}{\lambda}\right)^{\alpha(k-1)} \exp\left[-\alpha \left(\frac{x_i-\mu}{\lambda}\right)^k\right]
\end{eqnarray}

\end{document}

